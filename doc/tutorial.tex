%%%%%%%%%%%%%%%%%%%%%%%%%%%%%%%%%%%%%%%%%%%%%%%%%%%%%%%%%%%%%%%%%%%%%%%%%%%%%%%
\chapter{Tutorial}\label{chap:tuto}
%%%%%%%%%%%%%%%%%%%%%%%%%%%%%%%%%%%%%%%%%%%%%%%%%%%%%%%%%%%%%%%%%%%%%%%%%%%%%%%

The following chapter describes the usage of the PFFT library at the example of a simple test file in the first section,
followed by the more advanced features of PFFT in the next sections.

\section{A first parallel transform - Three-dimensional FFT with two-dimensional data decomposition}
We explain the basic steps for computing a parallel FFT with the PFFT library at the example
of the short test program given by Listing~\ref{lst:man_c2c}. This test computes a three-dimensional c2c-FFT on
a two-dimensional process mesh. The source code \code{manual_c2c_3d.c} can be found in directory \code{tests/}
of the library's source code tree. 
\lstinputlisting[numbers=left, float, caption={Minimal parallel c2c-FFT test program.}, label=lst:man_c2c]{../tests/manual_c2c_3d.c}

After initializing MPI with \code{MPI_Init} and before calling any other PFFT routine initialize
the parallel FFT computations via
\begin{lstlisting}
void pfft_init(void);
\end{lstlisting}
MPI introduces the concept of communicators to store all the topological information of the physical process layout.
PFFT requires to be called on a process mesh that corresponds to a periodic, Cartesian communicator.
We assist the user in creating such a communicator with the following routine
\begin{lstlisting}
int pfft_create_procmesh_2d(
    MPI_Comm comm, int np0, int np1,
    MPI_Comm *comm_cart_2d);
\end{lstlisting}
This routine uses the processes within the communicator \code{comm} to create a two-dimensional process
grid of size \code{np0} x \code{np1} and stores it into the Cartesian communicator \code{comm_cart_2d}.
Note that \code{comm_cart_2d} is allocated by the routine and must be freed with \code{MPI_Comm_free} after usage.
The input parameter \code{comm} is a communicator, indicating which processes will participate in the transform.
Choosing \code{comm} as \code{MPI_COMM_WORLD} implies that the FFT is computed on all available processes.

At the next step we need to know the data decomposition of the input and output array, that depends on
the array sizes, the process grid and the chosen parallel algorithm. Therefore, we call
\begin{lstlisting}
ptrdiff_t pfft_local_size_3d(
    ptrdiff_t *n, MPI_Comm comm_cart_2d, unsigned pfft_flags,
    ptrdiff_t *local_ni, ptrdiff_t *local_i_start,
    ptrdiff_t *local_no, ptrdiff_t *local_o_start);
\end{lstlisting}
Hereby, \code{n}, \code{local_ni}, \code{local_i_start}, \code{local_no}, \code{local_o_start} are arrays of length $3$ that must be allocated.
The return value of this function equals the size of the local complex array that needs to be allocated by every process.
In most cases, this coincides with the product of the local array sizes -- but may be bigger,
whenever the parallel algorithm needs some extra storage.
The input value \code{n} gives the three-dimensional FFT size and the flag \code{pfft_flags} serves to adjust
some details of the parallel execution. For the sake of simplicity, we restrict our self to the case
\code{pfft_flags=PFFT_TRANSPOSED_NONE} for a while and explain the more sophisticated flags at a later point.
The output arrays \code{local_ni} and \code{local_i_start} give the size and the offset of the local input array
that result from the parallel block distribution of the global input array, i.e.,
every process owns the input data \code{in[k[0],k[1],k[2]]} with \code{local_i_start[t] <= k[t] < local_i_start[t] + local_ni[t]}
for \code{t=0,1,2}. Analogously, the output parameters \code{local_o_start} and \code{local_no} contain the size
and the offset of the local output array.

Afterward, the input and output arrays must be allocated. Hereby,
\begin{lstlisting}
pfft_complex* pfft_alloc_complex(size_t size);
\end{lstlisting}
is a simple wrapper of \code{fftw_alloc_complex}, which in turn allocates the memory via \code{fftw_malloc} to ensure proper alignment for SIMD.
Have a look at the FFTW user manual~\cite{fftw-align-mem} for more details on SIMD memory alignment and \code{fftw_malloc}.
Nevertheless, you can also use any other dynamic memory allocation.

The planning of a single three-dimensional parallel FFT of size \code{n[0]} x \code{n[1]} x \code{n[2]}
is done by the function
\begin{lstlisting}
pfft_plan pfft_plan_dft_3d(
    ptrdiff_t *n, pfft_complex *in, pfft_complex *out,
    MPI_Comm comm_cart_2d, int sign, unsigned pfft_flags);
\end{lstlisting}
We provide the address of the input and output array by the pointers \code{in} and \code{out},
respectively. An inplace transform is assumed if these pointers are equal.
The integer \code{sign} gives the sign in the exponential of the FFT. Possible values are \code{PFFT_FORWARD} ($-1$)
and \code{PFFT_BACKWARD} ($+1$).
Flags passed to the planner via \code{pfft\_flags} must coincide with the flags that were passed to \code{pfft_local_size_3d}.
Otherwise the data layout of the parallel execution may not match calculated local array sizes.
As return value we get a PFFT plan, some structure that stores all the information needed to perform a parallel FFT.

Once the plan is generated, we are allowed to fill the input array \code{in}. Note, that per default the planning step
\code{pfft_plan_dft_3d} will overwrite input array \code{in}. Therefore, you should not write any sensitive data into \code{in} until the plan was generated.
For simplicity, our test program makes use of the library function
\begin{lstlisting}
void pfft_init_input_complex_3d(
    ptrdiff_t *n, ptrdiff_t *local_ni, ptrdiff_t *local_i_start,
    pfft_complex *in);
\end{lstlisting}
to fill the input array with some numbers. Alternatively, one can fill the array with a function \code{func} of choice
and the following loop that takes account of the parallel data layout
\begin{lstlisting}
ptrdiff_t m=0;
for(ptrdiff_t k0=0; k0 < local_ni[0]; k0++)
  for(ptrdiff_t k1=0; k1 < local_ni[1]; k1++)
    for(ptrdiff_t k2=0; k2 < local_ni[2]; k2++)
      in[m++] = func(k0 + local_i_start[0],
                     k1 + local_i_start[1],
                     k2 + local_i_start[2]);
\end{lstlisting}
The parallel FFT is computed when we execute the generated plan via
\begin{lstlisting}
void pfft_execute(const pfft_plan plan);
\end{lstlisting}
Now, the results can be read from \code{out} with an analogous three-dimensional loop.
If we do not want to execute another parallel FFT of the same type, we free the allocated memory of the plan with
\begin{lstlisting}
void pfft_destroy_plan(pfft_plan plan);
\end{lstlisting}
Additionally, we use
\begin{lstlisting}
int MPI_Comm_free(MPI_Comm *comm);  
\end{lstlisting}
to free the communicator allocated by \code{pfft_create_procmesh_2d} and
\begin{lstlisting}
void pfft_free(void *ptr);
\end{lstlisting}
to free memory allocated by \code{pfft_alloc_complex}.
Finally, we exit MPI via
\begin{lstlisting}
int MPI_Finalize(void);
\end{lstlisting}


\section{Porting FFTW-MPI based code to PFFT}\label{sec:porting}
We illustrate the close connection between FFTW-MPI and PFFT at a three-dimensional MPI example analogous to the example given in the FFTW manual~\cite{fftw-2dmpi}.
\begin{lstlisting}
#include <fftw3-mpi.h>
     
int main(int argc, char **argv)
{
    const ptrdiff_t n0 = ..., n1 = ..., n2 = ...;
    fftw_plan plan;
    fftw_complex *data;
    ptrdiff_t alloc_local, local_n0, local_0_start, i, j, k;

    MPI_Init(&argc, &argv);
    fftw_mpi_init();

    /* get local data size and allocate */
    alloc_local = fftw_mpi_local_size_3d(n0, n1, n2, MPI_COMM_WORLD,
					 &local_N0, &local_0_start);
    data = fftw_alloc_complex(alloc_local);

    /* create plan for in-place forward DFT */
    plan = fftw_mpi_plan_dft_3d(n0, n1, n2, data, data, MPI_COMM_WORLD,
				FFTW_FORWARD, FFTW_ESTIMATE);

    /* initialize data to some function my_function(x,y) */
    for (i = 0; i < local_n0; ++i) 
      for (j = 0; j < n1; ++j) 
        for (k = 0; k < n2; ++k)
          data[i*n1*n2 + j*n2 + k] = my_function(local_0_start + i, j, k);

    /* compute transforms, in-place, as many times as desired */
    fftw_execute(plan);

    fftw_destroy_plan(plan);

    MPI_Finalize();
}
\end{lstlisting}

Exactly the same task can be performed with PFFT as given in Listing~\ref{lst:pfft_3don1d}.
\begin{lstlisting}
#include <pfft.h>
     
int main(int argc, char **argv)
{
    const ptrdiff_t n[3] = {..., ..., ...};
    pfft_plan plan;
    pfft_complex *data;
    ptrdiff_t alloc_local, local_ni[3], local_i_start[3], local_no[3], local_o_start[3], i, j, k;
    unsigned pfft_flags = 0;

    MPI_Init(&argc, &argv);
    pfft_init();

    /* get local data size and allocate */
    alloc_local = pfft_local_size_dft_3d(n, MPI_COMM_WORLD, pfft_flags,
				         local_ni, local_i_start,
				         local_no, local_o_start);
    data = pfft_alloc_complex(alloc_local);

    /* create plan for in-place forward DFT */
    plan = pfft_plan_dft_3d(n, data, data, MPI_COMM_WORLD,
			    PFFT_FORWARD, PFFT_ESTIMATE);

    /* initialize data to some function my_function(x,y,z) */
    for (i = 0; i < local_n[0]; ++i) 
      for (j = 0; j < n[1]; ++j) 
        for (k = 0; k < n[2]; ++k)
          data[i*n[1]*n[2] + j*n[2] + k] = my_function(local_i_start[0] + i, j, k);

    /* compute transforms, in-place, as many times as desired */
    pfft_execute(plan);

    pfft_destroy_plan(plan);

    MPI_Finalize();
}
\end{lstlisting}



\begin{compactitem}
  \item substitute \code{fftw3-mpi.h} by \code{pfft.h}
  \item substitute all prefixes \code{fftw_} and \code{fftw_mpi_} by \code{pfft_}
  \item substitute all prefixes \code{FFTW_} by \code{PFFT_}
  \item the integers \code{N}, \code{local_n0}, \code{local_0_start} become arrays of length 3
  \item \code{dft_} in \code{pfft_local_size_dft_3d}
  \item \code{pfft_local_size_dft_3d} has additional input \code{pfft_flags} and additional outputs \code{local_no}, \code{local_o_start}
  \item The loop that inits \code{data} becomes splitted along all three dimensions. We could also use 
  
  
\end{compactitem}


First, All prefixes \code{fftw_} are substituted by \code{pfft_}

Now, the changes in order to use a two-dimensional process mesh are marginal as can be seen in Listing~\ref{lst:pfft_3don2d}.
\begin{lstlisting}
#include <pfft.h>
     
int main(int argc, char **argv)
{
    const ptrdiff_t n[3] = {..., ..., ...};
    (*@\color{red}const int np0 = ..., np1 = ...;@*)
    pfft_plan plan;
    pfft_complex *data;
    ptrdiff_t alloc_local, local_ni[3], local_i_start[3], local_no[3], local_o_start[3], i, j, k;
    unsigned pfft_flags = 0;
    (*@\color{red}MPI\_Comm comm\_cart\_2d;@*)

    MPI_Init(&argc, &argv);
    pfft_init();

    (*@\color{red}/* create two-dimensional process grid of size np0 x np1 */@*)
    (*@\color{red}pfft\_create\_procmesh\_2d(MPI\_COMM\_WORLD, np0, np1,@*)
        (*@\color{red}\&comm\_cart\_2d);@*)
    
    /* get local data size and allocate */
    alloc_local = pfft_local_size_dft_3d(n, (*@\color{red}comm\_cart\_2d@*), pfft_flags,
				         local_ni, local_i_start,
				         local_no, local_o_start);
    data = pfft_alloc_complex(alloc_local);

    /* create plan for in-place forward DFT */
    plan = pfft_plan_dft_3d(n, data, data, MPI_COMM_WORLD,
			    PFFT_FORWARD, PFFT_ESTIMATE);

    /* initialize data to some function my_function(x,y,z) */
    for (i = 0; i < local_n[0]; ++i) 
      for (j = 0; j < (*@\color{red}local\_n[1]@*); ++j) 
        for (k = 0; k < (*@\color{red}local\_n[2]@*); ++k)
          data[i*(*@\color{red}local\_n[1]*local\_n[2]@*) + j*(*@\color{red}local\_n[2]@*) + k] =
              my_function(local_i_start[0] + i,
		          (*@\color{red}local\_i\_start[1] +@*) j,
		          (*@\color{red}local\_i\_start[2] +@*) k);

    /* compute transforms, in-place, as many times as desired */
    pfft_execute(plan);

    pfft_destroy_plan(plan);

    MPI_Finalize();
}
\end{lstlisting}







\section{Errorcode for communicator creation}
As we have seen the function
\begin{lstlisting}
int pfft_create_procmesh_2d(
    MPI_Comm comm, int np0, int np1,
    MPI_Comm *comm_cart_2d);
\end{lstlisting}
creates a two-dimensional, periodic, Cartesian communicator. The \code{int} return value
(not used in Listing~\ref{lst:min_c2c}) is the forwarded error code of \code{MPI_Cart_create}.
It is equal to zero if the communicator was created successfully.
The most common error is that the number of processes within the input
communicator \code{comm} does not fit \code{np0 x np1}. In this case the Cartesian communicator
is not generated and the return value is unequal to zero. Therefore, a typical sanity check might look like
\begin{lstlisting}
/* Create two-dimensional process grid of size np[0] x np[1],
   if possible */
if( pfft_create_procmesh_2d(MPI_COMM_WORLD, np[0], np[1],
        &comm_cart_2d) )
{
  pfft_fprintf(MPI_COMM_WORLD, stderr,
      "Error: This test file only works with %d processes.\n",
      np[0]*np[1]);
  MPI_Finalize();
  return 1;
}
\end{lstlisting}
Hereby, we use the PFFT library function
\begin{lstlisting}
void pfft_fprintf(
    MPI_Comm comm, FILE *stream, const char *format, ...);
\end{lstlisting}
to print the error message.
This function is similar to the standard C function \code{fprintf} with the exception, that only the process with MPI rank $0$
within the given communicator \code{comm} will produce some output; see Section~\ref{sec:fprintf} for details.

\section{Inplace transforms}
Similar to FFTW, PFFT is able to compute parallel FFTs completely in place, which means that beside some
constant buffers, no second data array beside the inputs is necessary. Especially, the global data communication
can be performed in place. As far as we know, there is no other parallel FFT library beside FFTW and PFFT that
supports this feature. 
This feature is enabled as soon as the pointer to the output array \code{out} is equal to the pointer to the input array \code{in}.
E.g., in Listing~\ref{lst:min_c2c} we would call
\begin{lstlisting}[firstnumber=34]
/* Plan parallel forward FFT */
plan = pfft_plan_dft_3d(n, in, in, comm_cart_2d,
    PFFT_FORWARD, PFFT_TRANSPOSED_NONE);
\end{lstlisting}

\section{Higher dimensional data decomposition}
Our first test program used a two-dimensional data decomposition of a three-dimensional data set.
Moreover, PFFT support the computation of any $d$-dimensional FFT with $r$-dimensional data decomposition
as long as $r\le d-1$. For example, one can use a one-dimensional data decomposition for any two- or higher-dimensional data set,
while the data set must be at least four-dimensional to fit to a three-dimensional data decomposition.
The case $r=d$ is not supported efficiently, since during the parallel computations
there is always at least one dimension that remains local, i.e., one dimensions stays non-decomposed.
However, for the case $d=r=3$ we offer some library

The dimensionality of the data decomposition is given by the dimension of the Cartesian communicator that
goes into the PFFT planing routines. Therefore, we present a generalization of communicator creation function
\begin{lstlisting}
int pfft_create_procmesh(
    int rnk, MPI_Comm comm, const int *np,
    MPI_Comm *comm_cart);
\end{lstlisting}
Hereby, the array \code{np} of length \code{rnk} gives the size of the Cartesian communicator \code{cart_comm}.


\section{Transposed data layout}
\begin{compactitem}
  \item \code{PFFT_TRANSPOSED_NONE}
  \item \code{PFFT_TRANSPOSED_IN}
  \item \code{PFFT_TRANSPOSED_OUT}
\end{compactitem}

\subsection{Parallel data decomposition}

A $d$-dimensional array of size $n_0 \times n_1\times \hdots \times n_{d-1}$ 

Suppose a $r$-dimensional data decomposition
Suppose a $d$-dimensional array that is 
\code{PFFT_TRANSPOSED_NONE}
\begin{equation*}
  \frac{n_0}{P_0} \times \frac{n_1}{P_1} \times \hdots \times \frac{n_{r-1}}{P_{r-1}}  \times n_r \times n_{r+1} \times \hdots \times n_{d-1}
\end{equation*}



Transposed $d$-dimensional array distributed on $c$-dimensional process mesh ($c<d$)
\begin{equation*}
  \frac{n_1}{P_0} \times \frac{n_2}{P_1} \times \hdots \times \frac{n_c}{P_{c-1}}  \times n_0 \times n_{c+1} \times \hdots \times n_{d-1}
\end{equation*}

Comparision of non-transposed (left) and transposed (right) $d$-dimensional array for $c$-dimensional process mesh ($c<d$)
\begin{equation*}
  n_0\times n_1\times \hdots n_{c-1} \times n_c \hdots \times n_{d-1} \Rightarrow n_1 \times \hdots n_{c-1} \times n_0 \times n_c \times \hdots \times n_{d-1}
\end{equation*}

\section{Planning effort}
\begin{compactitem}
  \item \code{PFFT_ESTIMATE}
  \item \code{PFFT_MEASURE}
  \item \code{PFFT_PATIENT}
  \item \code{PFFT_EXHAUSIVE}
  \item \code{PFFT_NO_TUNE}
  \item \code{PFFT_TUNE}
\end{compactitem}


\section{Preserving input data}
\begin{compactitem}
  \item \code{PFFT_PRESERVE_INPUT}
  \item \code{PFFT_DESTROY_INPUT}
  \item \code{PFFT_BUFFERED_INPLACE}
\end{compactitem}

\section{FFTs with shifted index sets}
\begin{compactitem}
  \item \code{PFFT_SHIFTED_IN}
  \item \code{PFFT_SHIFTED_OUT}
\end{compactitem}

\section{PFFT flags}
\begin{compactitem}
  \item \code{PFFT_TRANSPOSED_NONE}
  \item \code{PFFT_TRANSPOSED_IN}
  \item \code{PFFT_TRANSPOSED_OUT}
  \item \code{PFFT_SHIFTED_IN}
  \item \code{PFFT_SHIFTED_OUT}
  \item \code{PFFT_ESTIMATE}
  \item \code{PFFT_MEASURE}
  \item \code{PFFT_PATIENT}
  \item \code{PFFT_EXHAUSIVE}
  \item \code{PFFT_NO_TUNE}
  \item \code{PFFT_TUNE}
  \item \code{PFFT_PRESERVE_INPUT}
  \item \code{PFFT_DESTROY_INPUT}
  \item \code{PFFT_BUFFERED_INPLACE}
\end{compactitem}


\section{Parallel data distribution}

\section{Three-dimensional FFTs with three-dimensional data decomposition}



\newpage
\section{Parallel FFT Frameworks}

\subsection{Forward Framework A (transposed input)}
\figurename{}~\ref{fig:fft_forw} lists the pseudo code of the parallel forward FFT framework.
\begin{figure}[ht]
  \begin{algorithmic}[1]
  %   \State\Comment{Calculate the serial FFTs row wise}
    \For{$t\gets0,\hdots,d-r-2$}
      \State $h_0 \gets \bigtimes_{s=0}^{r-1} N_s/P_s \times \bigtimes_{s=r}^{d-2-t} N_s$
      \State $N   \gets N_{d-1-t}$
      \State $h_1 \gets \bigtimes_{s=d-t}^{d-1} \hat N_s \times h$
      \State $h_0 \times N \times h_1 \osetarrow{FFT} h_0 \times \hat N \times h_1$
    \EndFor
    \For{$t\gets 0,\hdots,r-1$}
      \State $h_0 \gets \bigtimes_{s=r-t}^{r-1} \hat N_{s+1}/P_s \times \bigtimes_{s=0}^{r-t-1} N_s/P_s$
      \State $N   \gets N_{r-t}$
      \State $h_1 \gets \bigtimes_{s=r+1}^{d-1} \hat N_s \times h$
      \State $h_0 \times N \times h_1 \ousetarrow{FFT}{TO} \hat N \times h_0 \times h_1$
      \State
      \State $L_0 \gets \hat N_{r-t}$
      \State $h_0 \gets \bigtimes_{s=r-t}^{r-1}\hat N_{s+1}/P_{s} \times \bigtimes_{s=0}^{r-t-2} N_s/P_s$
      \State $L_1 \gets N_{r-t-1}$
      \State $h_1 \gets 1$
      \State $h_2 \gets \bigtimes_{s=r+1}^{d-1} \hat N_s \times h$
      \State $P   \gets P_{r-t-1}$
%       \State \mbox{$L_0 \times h_0 \times L_1/P \times h_1 \times h_2$} \mbox{$\ousetarrow{T}{TI} L_0/P \times h_0 \times L_1 \times h_1 \times h_2$}
      \State $L_0 \times h_0 \times L_1/P \times h_1 \times h_2\ousetarrow{T}{TI} L_0/P \times h_0 \times L_1 \times h_1 \times h_2$
    \EndFor
    \State $h_0 \gets \bigtimes_{s=0}^{r-1}\hat N_{s+1}/P_s$
    \State $N   \gets N_0$
    \State $h_1 \gets \bigtimes_{s=r+1}^{d-1} \hat N_s \times h$
    \State $h_0 \times N \times h_1 \osetarrow{FFT} h_0 \times \hat N \times h_1$
  \end{algorithmic}
  \caption{Parallel Forward FFT Framework}\label{fig:fft_forw}
\end{figure}

Within the first loop we use the serial FFT module~\eqref{eq:pruned_fft} to calculate
the one-dimensional (pruned) FFTs along the last $d-r-1$ array dimensions.
In the second loop we calculate $r$ one-dimensional pruned FFTs with transposed output~\eqref{eq:pruned_fft_to}
interleaved by global data transpositions with transposed input~\eqref{eq:gtransp_ti}.
Finally, a single non-transposed FFT~\eqref{eq:pruned_fft} must be computed to finish the full $d$-dimensional FFT.
The data decomposition of the output is then given by
\begin{equation*}
  \hat N_1/P_0 \times \hdots \times \hat N_{r-2}/P_{r-1} \times \hat N_r \times \hdots \times \hat N_{d-1} \times h\,.
\end{equation*}
Note, that the dimensions of the output array are slightly transposed.

\subsection{Backward Framework C (transposed output)}
Now, the parallel backward FFT framework can be derived very easy since we only need to revert all the steps
of the forward framework. The backward framework starts with the output decomposition of the forward framework
\begin{equation*}
  \hat N_1/P_0 \times \hdots \times \hat N_{r-2}/P_{r-1} \times \hat N_r \times \hdots \times \hat N_{d-1} \times h
\end{equation*}
and ends with the initial data decomposition
\begin{equation*}
  N_0/P_0 \times \hdots \times N_{r-1}/P_{r-1} \times N_r \times \hdots \times N_{d-1} \times h\,.
\end{equation*}
\figurename{}~\ref{fig:fft_back} lists the parallel backward FFT framework in pseudo code.
\begin{figure}[ht]
  \begin{algorithmic}[1]
  %   \State\Comment{Calculate the serial FFTs row wise}
    \State $h_0 \gets \bigtimes_{s=0}^{r-1}\hat N_{s+1}/P_s$
    \State $N   \gets \hat N_0$
    \State $h_1 \gets \bigtimes_{s=r+1}^{d-1} \hat N_s \times h$
    \State $h_0 \times \hat N \times h_1 \osetarrow{FFT} h_0 \times N \times h_1$
    \For{$t\gets r-1,\hdots,0$}
      \State $L_1 \gets \hat N_{r-t}$
      \State $h_1 \gets \bigtimes_{s=r-t}^{r-1}\hat N_{s+1}/P_{s} \times \bigtimes_{s=0}^{r-t-2} N_s/P_s$
      \State $L_0 \gets N_{r-t-1}$
      \State $h_0 \gets 1$
      \State $h_2 \gets \bigtimes_{s=r+1}^{d-1} \hat N_s \times h$
      \State $P   \gets P_{r-t-1}$
      \State $L_1/P \times h_1 \times L_0 \times h_0 \times h_2\ousetarrow{T}{TO} L_1 \times h_1 \times L_0/P \times h_0 \times h_2$
%       \State \mbox{$L_1/P \times h_1 \times L_0 \times h_0 \times h_2$} \mbox{$\ousetarrow{T}{TO} L_1 \times h_1 \times L_0/P \times h_0 \times h_2$}
      \State
      \State $h_0 \gets \bigtimes_{s=r-t}^{r-1} \hat N_{s+1}/P_s \times \bigtimes_{s=0}^{r-t-1} N_s/P_s$
      \State $N   \gets \hat N_{r-t}$
      \State $h_1 \gets \bigtimes_{s=r+1}^{d-1} \hat N_s \times h$
      \State $\hat N \times h_0 \times h_1 \ousetarrow{FFT}{TI} h_0 \times N \times h_1 $
    \EndFor
    \For{$t\gets d-r-2,\hdots,0$}
      \State $h_0 \gets \bigtimes_{s=0}^{r-1} N_s/P_s \times \bigtimes_{s=r}^{d-2-t} N_s$
      \State $N   \gets \hat N_{d-1-t}$
      \State $h_1 \gets \bigtimes_{s=d-t}^{d-1} \hat N_s \times h$
      \State $h_0 \times \hat N \times h_1 \osetarrow{FFT} h_0 \times N \times h_1$
    \EndFor
  \end{algorithmic}
  \caption{Parallel Backward FFT Framework}\label{fig:fft_back}
\end{figure}

\newpage
\subsection{Backward Framework A (transposed input) - General \texorpdfstring{$r$, $d$}{r, d}}
\setlength{\arraycolsep}{2pt}
\begin{equation*}
  \begin{array}{cc}
    & \frac{\hat N_1}{P_0} \times \hdots \times \frac{\hat N_r}{P_{r-1}} \times \hat N_0 \times \hat N_{r+1} \times \hdots \times \hat N_{d-1} \times h \\
    \ousetarrow{FFT}{TO} & N_0 \times \frac{\hat N_1}{P_0} \times \hdots \times \frac{\hat N_r}{P_{r-1}} \times \hat N_{r+1} \times \hdots \times \hat N_{d-1} \times h \\
    \transposearrow{TI} & \frac{N_0}{P_0} \times \hat N_1 \times \frac{\hat N_2}{P_1} \times \hdots \times \frac{\hat N_r}{P_{r-1}} \times \hat N_{r+1} \times \hdots \times \hat N_{d-1} \times h \\
    \ousetarrow{FFT}{TO} & N_1 \times \frac{N_0}{P_0} \times \frac{\hat N_2}{P_1} \times \hdots \times \frac{\hat N_r}{P_{r-1}} \times \hat N_{r+1} \times \hdots \times \hat N_{d-1} \times h \\
    \vdots & \\
    \transposearrow{TI} & \frac{N_{r-1}}{P_{r-1}} \times \hdots \times \frac{N_0}{P_0} \times \hat N_r \times \hat N_{r+1} \times \hdots \times \hat N_{d-1} \times h \\
    \ousetarrow{FFT}{\red{T}} & \frac{N_0}{P_0} \times \hdots \times \frac{N_{r-1}}{P_{r-1}} \times N_r \times \hat N_{r+1} \times \hdots \times \hat N_{d-1} \times h \\
    \ousetarrow{FFT}{} & \frac{N_0}{P_0} \times \hdots \times \frac{N_{r-1}}{P_{r-1}} \times N_r \times N_{r+1} \times \hat N_{r+2} \times \hdots \times \hat N_{d-1} \times h \\
    \vdots & \\
    \ousetarrow{FFT}{} & \frac{N_0}{P_0} \times \hdots \times \frac{N_{r-1}}{P_{r-1}} \times N_r \times N_{r+1} \times \hdots \times N_{d-1} \times h \\
  \end{array}
\end{equation*}

\newpage
\subsection{Backward Framework A (transposed input) for \texorpdfstring{$r=3$, $d=5$}{r=3 and d=5}}
\setlength{\arraycolsep}{2pt}
\begin{equation*}
  \begin{array}{cccc}
    & \left(\frac{\hat N_1}{P_0} \times \frac{\hat N_2}{P_1} \times \frac{\hat N_3}{P_2}\right) \times \hat N_0 \times \left(\hat N_4 \times h\right)
    & \ousetarrow{FFT}{TO} & \Big(N_0\Big) \times \left(\frac{\hat N_1}{P_0} \times \frac{\hat N_2}{P_1} \times \frac{\hat N_3}{P_2}\right) \times \left(\hat N_4 \times h\right) \\
    \transposearrow{TI} & \left(\frac{N_0}{P_0}\right) \times \hat N_1 \times \left(\frac{\hat N_2}{P_1} \times \frac{\hat N_3}{P_2} \times \hat N_4 \times h\right)
    & \ousetarrow{FFT}{TO} & \left(N_1 \times \frac{N_0}{P_0}\right) \times \left(\frac{\hat N_2}{P_1} \times \frac{\hat N_3}{P_2} \right) \times \left(\hat N_4 \times h\right) \\
    \transposearrow{TI} & \left(\frac{N_1}{P_1} \times \frac{N_0}{P_0}\right) \times \hat N_2 \times \left(\frac{\hat N_3}{P_2} \times \hat N_4 \times h \right)
    & \ousetarrow{FFT}{TO} & \left(N_2 \times \frac{N_1}{P_1} \times \frac{N_0}{P_0}\right) \times \left(\frac{\hat N_3}{P_2} \right) \times \left(\hat N_4 \times h\right) \\
    \transposearrow{TI} & \frac{N_2}{P_2} \times \frac{N_1}{P_1} \times \frac{N_0}{P_0} \times \left(\hat N_3 \times \hat N_4 \times h \right)
    & \ousetarrow{FFT}{\red{T}} & \frac{N_0}{P_0} \times \frac{N_1}{P_1} \times \frac{N_2}{P_2} \times N_3 \times N_4 \times h
  \end{array}
\end{equation*}
Last step need more general interface to serial FFT module!

\subsection{Backward Framework B (transposed input and output) for \texorpdfstring{$r=3$, $d=5$}{r=3 and d=5}}
\setlength{\arraycolsep}{2pt}
\begin{equation*}
  \begin{array}{cccc}
    & \left(\frac{\hat N_1}{P_0} \times \frac{\hat N_2}{P_1} \times \frac{\hat N_3}{P_2}\right) \times \hat N_0 \times \left(\hat N_4 \times h\right)
    & \ousetarrow{FFT}{TO} & \Big(N_0\Big) \times \left(\frac{\hat N_1}{P_0} \times \frac{\hat N_2}{P_1} \times \frac{\hat N_3}{P_2}\right) \times \left(\hat N_4 \times h\right) \\
    \transposearrow{TIO} & \hat N_1 \times \left(\frac{\hat N_2}{P_1} \times \frac{\hat N_3}{P_2} \times \frac{N_0}{P_0}\right) \times \left(\hat N_4 \times h\right)
    & \osetarrow{FFT} & \Big(N_1\Big) \times \left(\frac{\hat N_2}{P_1} \times \frac{\hat N_3}{P_2} \times \frac{N_0}{P_0}\right) \times \left(\hat N_4 \times h\right) \\
    \transposearrow{TIO} & \hat N_2 \times \left(\frac{\hat N_3}{P_2} \times \frac{N_0}{P_0} \times \frac{N_1}{P_1}\right) \times \left(\hat N_4 \times h\right)
    & \osetarrow{FFT} & \Big(N_2\Big) \times \left(\frac{\hat N_3}{P_2} \times \frac{N_0}{P_0} \times \frac{N_1}{P_1}\right) \times \left(\hat N_4 \times h\right) \\
    \transposearrow{TIO} & \hat N_3 \times \left(\frac{N_0}{P_0} \times \frac{N_1}{P_1}  \times \frac{N_2}{P_2}\right) \times \left(\hat N_4 \times h\right)
    & \ousetarrow{FFT}{TI} & \frac{N_0}{P_0} \times \frac{N_1}{P_1}  \times \frac{N_2}{P_2} \times N_3 \times N_4 \times h
  \end{array}
\end{equation*}

\subsection{Forward Framework A (transposed input) for \texorpdfstring{$r=3$, $d=5$}{r=3 and d=5}}
\setlength{\arraycolsep}{2pt}
\begin{equation*}
  \begin{array}{cccc}
    & \left(\frac{N_0}{P_0} \times \frac{N_1}{P_1}  \times \frac{N_2}{P_2}\right) \times N_3 \times \left(\hat N_4 \times h\right)
    & \ousetarrow{FFT}{TO} & \left(\hat N_3 \times \frac{N_0}{P_0} \times \frac{N_1}{P_1}\right) \times \left(\frac{N_2}{P_2}\right) \times \left(\hat N_4 \times h\right) \\
    \transposearrow{TI} & \left(\frac{\hat N_3}{P_2} \times \frac{N_0}{P_0} \times \frac{N_1}{P_1}\right)  \times N_2 \times \left(\hat N_4 \times h\right)
    & \ousetarrow{FFT}{TO} & \left(\hat N_2 \times \frac{\hat N_3}{P_2} \times \frac{N_0}{P_0}\right) \times \left(\frac{N_1}{P_1}\right) \times \left(\hat N_4 \times h\right) \\
    \transposearrow{TI} & \left(\frac{\hat N_2}{P_1} \times \frac{\hat N_3}{P_2} \times \frac{N_0}{P_0}\right) \times N_1 \times \left(\hat N_4 \times h\right)
    & \ousetarrow{FFT}{TO} & \left(\hat N_1 \times \frac{\hat N_2}{P_1} \times \frac{\hat N_3}{P_2}\right) \times \left(\frac{N_0}{P_0}\right) \times \left(\hat N_4 \times h\right) \\
    \transposearrow{TI} & \left(\frac{\hat N_1}{P_0} \times \frac{\hat N_2}{P_1} \times \frac{\hat N_3}{P_2}\right) \times N_0 \times \left(\hat N_4 \times h\right)
    & \osetarrow{FFT} & \frac{\hat N_1}{P_0} \times \frac{\hat N_2}{P_1} \times \frac{\hat N_3}{P_2} \times \hat N_0 \times \hat N_4 \times h
  \end{array}
\end{equation*}

\subsection{Forward Framework B (transposed input and output) for \texorpdfstring{$r=3$, $d=5$}{r=3 and d=5}}
\setlength{\arraycolsep}{2pt}
\begin{equation*}
  \begin{array}{cccc}
    & \left(\frac{N_0}{P_0} \times \frac{N_1}{P_1}  \times \frac{N_2}{P_2}\right) \times N_3 \times \left(N_4 \times h\right)
    & \ousetarrow{FFT}{TO} & \left(\hat N_3 \times \frac{N_0}{P_0} \times \frac{N_1}{P_1}\right)  \times \left(\frac{N_2}{P_2}\right) \times \left(\hat N_4 \times h\right) \\
    \transposearrow{TIO} & N_2 \times \left(\frac{\hat N_3}{P_2} \times \frac{N_0}{P_0} \times \frac{N_1}{P_1} \right) \times \left( \hat N_4 \times h\right)
    & \osetarrow{FFT} & \left(\hat N_2 \times \frac{\hat N_3}{P_2} \times \frac{N_0}{P_0}\right) \times \left(\frac{N_1}{P_1}\right) \times \left(\hat N_4 \times h\right) \\
    \transposearrow{TIO} & N_1 \times \left(\frac{\hat N_2}{P_1} \times \frac{\hat N_3}{P_2} \times \frac{N_0}{P_0}\right) \times \left( \hat N_4 \times h\right)
    & \osetarrow{FFT} & \left(\hat N_1 \times \frac{\hat N_2}{P_1} \times \frac{\hat N_3}{P_2}\right) \times \left(\frac{N_0}{P_0}\right) \times \left(\hat N_4 \times h\right) \\
    \transposearrow{TIO} & N_0 \times \left(\frac{\hat N_1}{P_0} \times \frac{\hat N_2}{P_1} \times \frac{\hat N_3}{P_2}\right) \times \left(\hat N_4 \times h\right)
    & \ousetarrow{FFT}{TI} & \frac{\hat N_1}{P_0} \times \frac{\hat N_2}{P_1} \times \frac{\hat N_3}{P_2} \times \hat N_0 \times \hat N_4 \times h
  \end{array}
\end{equation*}

\newpage
\subsection{Backward Framework A (transposed input) - The general form}
\figurename{}~\ref{fig:fft_back_A} lists the parallel backward FFT framework A in pseudo code.
\begin{figure}[ht]
  \begin{algorithmic}[1]
  %   \State\Comment{Calculate the serial FFTs row wise}
    \For{$t\gets 0,\hdots,r-1$}
      \State $h_0 \gets \bigtimes_{s=t}^{r-1}\hat N_{s+1}/P_s$
      \State $N   \gets \hat N_t$
      \State $h_1 \gets \bigtimes_{s=t+1}^{r-1} \hat N_{s+1}/ P_{s} \times \hat N_{r}  \bigtimes_{s=r+1}^{d-1} \hat N_s \times h$
      \State $h_0 \times \hat N \times h_1 \ousetarrow{FFT}{TO} N \times h_0 \times h_1$
      \State
      \State $L_1 \gets N_t$
      \State $h_1 \gets \bigtimes_{s=0}^{t-1} N_{s}/P_{s}$
      \State $L_0 \gets \hat N_{t+1}$
      \State $h_0 \gets \bigtimes_{s=t+1}^{r-1}\hat N_{s+1}/P_{s}$
      \State $h_2 \gets \bigtimes_{s=r+1}^{d-1} \hat N_s \times h$
      \State $P   \gets P_{t}$
      \State $L_1 \times h_1 \times L_0/P \times h_0 \times h_2\ousetarrow{T}{TI} L_1/P \times h_1 \times L_0 \times h_0 \times h_2$
    \EndFor
    \State $h_0 \gets \bigtimes_{s=r-t}^{r-1} \hat N_{s+1}/P_s \times \bigtimes_{s=0}^{r-t-1} N_s/P_s$
    \State $N   \gets \hat N_{r-t}$
    \State $h_1 \gets \bigtimes_{s=r+1}^{d-1} \hat N_s \times h$
    \State $\hat N \times h_0 \times h_1 \ousetarrow{FFT}{TI} h_0 \times N \times h_1 $
    \For{$t\gets d-r-2,\hdots,0$}
      \State $h_0 \gets \bigtimes_{s=0}^{r-1} N_s/P_s \times \bigtimes_{s=r}^{d-2-t} N_s$
      \State $N   \gets \hat N_{d-1-t}$
      \State $h_1 \gets \bigtimes_{s=d-t}^{d-1} \hat N_s \times h$
      \State $h_0 \times \hat N \times h_1 \osetarrow{FFT} h_0 \times N \times h_1$
    \EndFor
  \end{algorithmic}
  \caption{Parallel Backward FFT Framework A}\label{fig:fft_back_A}
\end{figure}


\begin{compactitem}
  \item[\mybox] \verb+#include <complex.h> #include <pfft.h>+
\end{compactitem}


\section{Precisions}\label{sec:prec}
This section an analog to part~\cite{fftw-prec} of the FFTW manual.

You can install single and long-double precision versions of PFFT, which replace double with float and long double, respectively; see \ref{sec:install}.
To use these interfaces, you must
\begin{compactitem}
  \item Link to the single/long-double libraries; on Unix, \code{-lpfftf} or \code{-lpfftl} instead of (or in addition to) \code{-lpfft}.
        (You can link to the different-precision libraries simultaneously.)
  \item Include the same \code{<pfft.h>} header file.
  \item Replace all lowercase instances of ‘\code{pfft_}’ with ‘\code{pfftf_}’ or ‘\code{pfftl_}’ for single or long-double precision, respectively.
        (\code{pfft_complex} becomes \code{pfftf_complex}, \code{pfft_execute} becomes \code{pfftf_execute}, etcetera.)
  \item Uppercase names, i.e. names beginning with ‘\code{PFFT_}’, remain the same.
  \item Replace double with float or long double for subroutine parameters.
\end{compactitem}

\section{Complex numbers}
PFFT introduces the complex data type \code{pfft_complex} that is nothing else than a typedef to \code{fftw_complex}.
According to the FFTW manual~\cite{fftw-cplx-num}:
\begin{quote}
  The default FFTW interface uses double precision for all floating-point numbers, and defines a \code{fftw_complex} type to hold complex numbers as:
  \begin{lstlisting}
    typedef double fftw_complex[2];
  \end{lstlisting}
  Here, the \code{[0]} element holds the real part and the \code{[1]} element holds the imaginary part.

  Alternatively, if you have a C compiler (such as gcc) that supports the C99 revision of the ANSI C standard,
  you can use C's new native complex type (which is binary-compatible with the typedef above).
  In particular, if you \code{#include <complex.h>} before \code{<fftw.h>}, then \code{fftw_complex} is defined to be the native complex
  type and you can manipulate it with ordinary arithmetic (e.g. x = y * (3+4*I), where x and y are \code{fftw_complex}
  and I is the standard symbol for the imaginary unit);
\end{quote}
I.e., include \code{<complex.h>} \emph{before} \code{<pfft.h>}, \code{<fftw.h>} and \code{<fftw-mpi.h>} to be sure that \code{pfft_complex} is defined to be the native complex.

\section{Measuring parallel run times}
Use \code{MPI_Barrier} in front of every call to \code{pfft_} function to avoid unbalanced run times.
