%%%%%%%%%%%%%%%%%%%%%%%%%%%%%%%%%%%%%%%%%%%%%%%%%%%%%%%%%%%%%%%%%%%%%%%%%%%%%%%
\chapter{PFFT reference}\label{chap:ref}
%%%%%%%%%%%%%%%%%%%%%%%%%%%%%%%%%%%%%%%%%%%%%%%%%%%%%%%%%%%%%%%%%%%%%%%%%%%%%%%

\section{Data Types and Files}
\subsection{Complex numbers}

\subsection{Precision}

\subsection{Memory allocation}


\section{Using plans}
\begin{lstlisting}
void pfft_execute(const pfft_plan plan);
void pfft_execute_dft(const pfft_plan plan, pfft_complex *in, pfft_complex *out);
void pfft_execute_dft_r2c(const pfft_plan plan, double *in, pfft_complex *out);
void pfft_execute_dft_c2r(const pfft_plan plan, pfft_complex *in, double *out);
void pfft_execute_r2r(const pfft_plan plan, double *in, double *out);
void pfft_destroy_plan(const pfft_plan plan);
\end{lstlisting}



\section{Complex-to-complex FFT}
\begin{lstlisting}
ptrdiff_t pfft_local_size_dft_3d(
    const ptrdiff_t *n, MPI_Comm comm_cart, unsigned pfft_flags,
    ptrdiff_t *local_ni, ptrdiff_t *local_i_start,
    ptrdiff_t *local_no, ptrdiff_t *local_o_start);
ptrdiff_t pfft_local_size_dft(
    int rnk, const ptrdiff_t *n,
    MPI_Comm comm_cart, unsigned pfft_flags,
    ptrdiff_t *local_ni, ptrdiff_t *local_i_start,
    ptrdiff_t *local_no, ptrdiff_t *local_o_start);
ptrdiff_t pfft_local_size_many_dft(
    int rnk, const ptrdiff_t *n, const ptrdiff_t *ni, const ptrdiff_t *no,
    ptrdiff_t howmany, const ptrdiff_t *iblock, const ptrdiff_t *oblock,
    MPI_Comm comm_cart, unsigned pfft_flags,
    ptrdiff_t *local_ni, ptrdiff_t *local_i_start,
    ptrdiff_t *local_no, ptrdiff_t *local_o_start);
\end{lstlisting}

\begin{lstlisting}
void pfft_local_block_dft_3d(
    const ptrdiff_t *n, MPI_Comm comm_cart, int pid, unsigned pfft_flags,
    ptrdiff_t *local_ni, ptrdiff_t *local_i_start,
    ptrdiff_t *local_no, ptrdiff_t *local_o_start);
void pfft_local_block_dft(
    int rnk, const ptrdiff_t *n,
    MPI_Comm comm_cart, int pid, unsigned pfft_flags,
    ptrdiff_t *local_ni, ptrdiff_t *local_i_start,
    ptrdiff_t *local_no, ptrdiff_t *local_o_start);
void pfft_local_block_many_dft(
    int rnk_n, const ptrdiff_t *ni, const ptrdiff_t *no,
    const ptrdiff_t *iblock, const ptrdiff_t *oblock,
    MPI_Comm comm_cart, int pid, unsigned pfft_flags,
    ptrdiff_t *local_ni, ptrdiff_t *local_i_start,
    ptrdiff_t *local_no, ptrdiff_t *local_o_start);
\end{lstlisting}

\begin{lstlisting}
pfft_plan pfft_plan_dft_3d(
    const ptrdiff_t *n, pfft_complex *in, pfft_complex *out, MPI_Comm comm_cart,
    int sign, unsigned pfft_flags);
pfft_plan pfft_plan_dft(
    int rnk, const ptrdiff_t *n, pfft_complex *in, pfft_complex *out, MPI_Comm comm_cart,
    int sign, unsigned pfft_flags);
pfft_plan pfft_plan_many_dft(
    int rnk, const ptrdiff_t *n, const ptrdiff_t *ni, const ptrdiff_t *no,
    ptrdiff_t howmany, const ptrdiff_t *iblock, const ptrdiff_t *oblock,
    pfft_complex *in, pfft_complex *out, MPI_Comm comm_cart,
    int sign, unsigned pfft_flags);
pfft_plan pfft_plan_many_dft_skipped(
    int rnk, const ptrdiff_t *n, const ptrdiff_t *ni, const ptrdiff_t *no,
    ptrdiff_t howmany, const ptrdiff_t *iblock, const ptrdiff_t *oblock,
    const int *skip_trafos, pfft_complex *in, pfft_complex *out, MPI_Comm comm_cart,
    int sign, unsigned pfft_flags);
\end{lstlisting}

\section{Real-to-complex FFT}
\begin{lstlisting}
ptrdiff_t pfft_local_size_dft_r2c_3d(
    const ptrdiff_t *n, MPI_Comm comm_cart, unsigned pfft_flags,
    ptrdiff_t *local_ni, ptrdiff_t *local_i_start,
    ptrdiff_t *local_no, ptrdiff_t *local_o_start);
ptrdiff_t pfft_local_size_dft_r2c(
    int rnk, const ptrdiff_t *n,
    MPI_Comm comm_cart, unsigned pfft_flags,
    ptrdiff_t *local_ni, ptrdiff_t *local_i_start,
    ptrdiff_t *local_no, ptrdiff_t *local_o_start);
ptrdiff_t pfft_local_size_many_dft_r2c(
    int rnk_n, const ptrdiff_t *n, const ptrdiff_t *ni, const ptrdiff_t *no,
    ptrdiff_t howmany, const ptrdiff_t *iblock, const ptrdiff_t *oblock,
    MPI_Comm comm_cart, unsigned pfft_flags,
    ptrdiff_t *local_ni, ptrdiff_t *local_i_start,
    ptrdiff_t *local_no, ptrdiff_t *local_o_start);
\end{lstlisting}

\begin{lstlisting}
void pfft_local_block_dft_r2c_3d(
    const ptrdiff_t *n, MPI_Comm comm_cart, int pid, unsigned pfft_flags,
    ptrdiff_t *local_ni, ptrdiff_t *local_i_start,
    ptrdiff_t *local_no, ptrdiff_t *local_o_start);
void pfft_local_block_dft_r2c(
    int rnk, const ptrdiff_t *n,
    MPI_Comm comm_cart, int pid, unsigned pfft_flags,
    ptrdiff_t *local_ni, ptrdiff_t *local_i_start,
    ptrdiff_t *local_no, ptrdiff_t *local_o_start);
void pfft_local_block_many_dft_r2c(
    int rnk_n, const ptrdiff_t *ni, const ptrdiff_t *no,
    const ptrdiff_t *iblock, const ptrdiff_t *oblock,
    MPI_Comm comm_cart, int pid, unsigned pfft_flags,
    ptrdiff_t *local_ni, ptrdiff_t *local_i_start,
    ptrdiff_t *local_no, ptrdiff_t *local_o_start);
\end{lstlisting}

\begin{lstlisting}
pfft_plan pfft_plan_dft_r2c_3d(
    const ptrdiff_t *n, double *in, pfft_complex *out, MPI_Comm comm_cart,
    int sign, unsigned pfft_flags);
pfft_plan pfft_plan_dft_r2c(
    int rnk_n, const ptrdiff_t *n, double *in, pfft_complex *out, MPI_Comm comm_cart,
    int sign, unsigned pfft_flags);
pfft_plan pfft_plan_many_dft_r2c(
    int rnk_n, const ptrdiff_t *n, const ptrdiff_t *ni, const ptrdiff_t *no,
    ptrdiff_t howmany, const ptrdiff_t *iblock, const ptrdiff_t *oblock,
    double *in, pfft_complex *out, MPI_Comm comm_cart,
    int sign, unsigned pfft_flags);
pfft_plan pfft_plan_many_dft_r2c_skipped(
    int rnk_n, const ptrdiff_t *n, const ptrdiff_t *ni, const ptrdiff_t *no,
    ptrdiff_t howmany, const ptrdiff_t *iblock, const ptrdiff_t *oblock,
    const int *skip_trafos, double *in, pfft_complex *out, MPI_Comm comm_cart,
    int sign, unsigned pfft_flags);
\end{lstlisting}

\section{Complex-to-real FFT}
\begin{lstlisting}
ptrdiff_t pfft_local_size_dft_c2r_3d(
    const ptrdiff_t *n, MPI_Comm comm_cart, unsigned pfft_flags,
    ptrdiff_t *local_ni, ptrdiff_t *local_i_start,
    ptrdiff_t *local_no, ptrdiff_t *local_o_start);
ptrdiff_t pfft_local_size_dft_c2r(
    int rnk, const ptrdiff_t *n,
    MPI_Comm comm_cart, unsigned pfft_flags,
    ptrdiff_t *local_ni, ptrdiff_t *local_i_start,
    ptrdiff_t *local_no, ptrdiff_t *local_o_start);
ptrdiff_t pfft_local_size_many_dft_c2r(
    int rnk_n, const ptrdiff_t *n, const ptrdiff_t *ni, const ptrdiff_t *no,
    ptrdiff_t howmany, const ptrdiff_t *iblock, const ptrdiff_t *oblock,
    MPI_Comm comm_cart, unsigned pfft_flags,
    ptrdiff_t *local_ni, ptrdiff_t *local_i_start,
    ptrdiff_t *local_no, ptrdiff_t *local_o_start);
\end{lstlisting}

\begin{lstlisting}
void pfft_local_block_dft_c2r_3d(
    const ptrdiff_t *n, MPI_Comm comm_cart, int pid, unsigned pfft_flags,
    ptrdiff_t *local_ni, ptrdiff_t *local_i_start,
    ptrdiff_t *local_no, ptrdiff_t *local_o_start);
void pfft_local_block_dft_c2r(
    int rnk, const ptrdiff_t *n,
    MPI_Comm comm_cart, int pid, unsigned pfft_flags,
    ptrdiff_t *local_ni, ptrdiff_t *local_i_start,
    ptrdiff_t *local_no, ptrdiff_t *local_o_start);
void pfft_local_block_many_dft_c2r(
    int rnk_n, const ptrdiff_t *ni, const ptrdiff_t *no,
    const ptrdiff_t *iblock, const ptrdiff_t *oblock,
    MPI_Comm comm_cart, int pid, unsigned pfft_flags,
    ptrdiff_t *local_ni, ptrdiff_t *local_i_start,
    ptrdiff_t *local_no, ptrdiff_t *local_o_start);
\end{lstlisting}

\begin{lstlisting}
pfft_plan pfft_plan_dft_c2r_3d(
    const ptrdiff_t *n, pfft_complex *in, double *out, MPI_Comm comm_cart,
    int sign, unsigned pfft_flags);
pfft_plan pfft_plan_dft_c2r(
    int rnk_n, const ptrdiff_t *n, pfft_complex *in, double *out, MPI_Comm comm_cart,
    int sign, unsigned pfft_flags);
pfft_plan pfft_plan_many_dft_c2r(
    int rnk_n, const ptrdiff_t *n, const ptrdiff_t *ni, const ptrdiff_t *no,
    ptrdiff_t howmany, const ptrdiff_t *iblock, const ptrdiff_t *oblock,
    pfft_complex *in, double *out, MPI_Comm comm_cart,
    int sign, unsigned pfft_flags);
pfft_plan pfft_plan_many_dft_c2r_skipped(
    int rnk_n, const ptrdiff_t *n, const ptrdiff_t *ni, const ptrdiff_t *no,
    ptrdiff_t howmany, const ptrdiff_t *iblock, const ptrdiff_t *oblock,
    const int *skip_trafos, pfft_complex *in, double *out, MPI_Comm comm_cart,
    int sign, unsigned pfft_flags);
\end{lstlisting}

\section{Real-to-real FFT}
\begin{lstlisting}
ptrdiff_t pfft_local_size_r2r_3d(
    const ptrdiff_t *n, MPI_Comm comm_cart, unsigned pfft_flags,
    ptrdiff_t *local_ni, ptrdiff_t *local_i_start,
    ptrdiff_t *local_no, ptrdiff_t *local_o_start);
ptrdiff_t pfft_local_size_r2r(
    int rnk_n, const ptrdiff_t *n,
    MPI_Comm comm_cart, unsigned pfft_flags,
    ptrdiff_t *local_ni, ptrdiff_t *local_i_start,
    ptrdiff_t *local_no, ptrdiff_t *local_o_start);
ptrdiff_t pfft_local_size_many_r2r(
    int rnk_n, const ptrdiff_t *n, const ptrdiff_t *ni, const ptrdiff_t *no,
    ptrdiff_t howmany, const ptrdiff_t *iblock, const ptrdiff_t *oblock,
    MPI_Comm comm_cart, unsigned pfft_flags,
    ptrdiff_t *local_ni, ptrdiff_t *local_i_start,
    ptrdiff_t *local_no, ptrdiff_t *local_o_start);
\end{lstlisting}

\begin{lstlisting}
void pfft_local_block_r2r_3d(
    const ptrdiff_t *n, MPI_Comm comm_cart, int pid, unsigned pfft_flags,
    ptrdiff_t *local_ni, ptrdiff_t *local_i_start,
    ptrdiff_t *local_no, ptrdiff_t *local_o_start);
void pfft_local_block_r2r(
    int rnk, const ptrdiff_t *n,
    MPI_Comm comm_cart, int pid, unsigned pfft_flags,
    ptrdiff_t *local_ni, ptrdiff_t *local_i_start,
    ptrdiff_t *local_no, ptrdiff_t *local_o_start);
void pfft_local_block_many_r2r(
    int rnk_n, const ptrdiff_t *ni, const ptrdiff_t *no,
    const ptrdiff_t *iblock, const ptrdiff_t *oblock,
    MPI_Comm comm_cart, int pid, unsigned pfft_flags,
    ptrdiff_t *local_ni, ptrdiff_t *local_i_start,
    ptrdiff_t *local_no, ptrdiff_t *local_o_start);
\end{lstlisting}

\begin{lstlisting}
pfft_plan pfft_plan_r2r_3d(
    const ptrdiff_t *n, double *in, double *out, MPI_Comm comm_cart,
    const pfft_r2r_kind *kinds, unsigned pfft_flags);
pfft_plan pfft_plan_r2r(
    int rnk_n, const ptrdiff_t *n, double *in, double *out, MPI_Comm comm_cart,
    const pfft_r2r_kind *kinds, unsigned pfft_flags);
pfft_plan pfft_plan_many_r2r(
    int rnk_n, const ptrdiff_t *n, const ptrdiff_t *ni, const ptrdiff_t *no,
    ptrdiff_t howmany, const ptrdiff_t *iblock, const ptrdiff_t *oblock,
    double *in, double *out, MPI_Comm comm_cart,
    const pfft_r2r_kind *kinds, unsigned pfft_flags);
pfft_plan pfft_plan_many_r2r_skipped(
    int rnk_n, const ptrdiff_t *n, const ptrdiff_t *ni, const ptrdiff_t *no,
    ptrdiff_t howmany, const ptrdiff_t *iblock, const ptrdiff_t *oblock,
    const int *skip_trafos, double *in, double *out, MPI_Comm comm_cart,
    const pfft_r2r_kind *kinds, unsigned pfft_flags);
\end{lstlisting}

\subsection{Timers for FFT computations}
\begin{lstlisting}
void pfft_reset_timer(
    pfft_plan ths);
pfft_timer pfft_get_timer(
    const pfft_plan ths);
void pfft_print_average_timer(
    const pfft_plan ths, MPI_Comm comm);
void pfft_print_average_timer_adv(
    const pfft_plan ths, MPI_Comm comm);
void pfft_write_average_timer(
    const pfft_plan ths, const char *name, MPI_Comm comm);
void pfft_write_average_timer_adv(
    const pfft_plan ths, const char *name, MPI_Comm comm);

pfft_timer pfft_copy_timer(
    const pfft_timer orig);
void pfft_average_timer(
    pfft_timer ths);
pfft_timer pfft_add_timers(
    const pfft_timer sum1, const pfft_timer sum2);
pfft_timer pfft_reduce_max_timer(
    const pfft_timer ths, MPI_Comm comm);
double* pfft_convert_timer2vec(
    const pfft_timer ths);
pfft_timer pfft_convert_vec2timer(
    const double *times);
void pfft_destroy_timer(
    pfft_timer ths);
\end{lstlisting}




\section{Ghost cell communication}

\begin{lstlisting}
void pfft_exchange(
    pfft_gcplan ths);
void pfft_reduce(
    pfft_gcplan ths);
void pfft_destroy_gcplan(
    pfft_gcplan ths);
\end{lstlisting}

\subsection{Local array size}\label{sec:gc:local_size}
\begin{lstlisting}
ptrdiff_t pfft_local_size_gc_3d(
    const ptrdiff_t *local_n, const ptrdiff_t *local_start,
    ptrdiff_t alloc_local, const ptrdiff_t *gc_below, const ptrdiff_t *gc_above,
    ptrdiff_t *local_ngc, ptrdiff_t *local_gc_start);
ptrdiff_t pfft_local_size_gc(
    int rnk_n, const ptrdiff_t *local_n, const ptrdiff_t *local_start,
    ptrdiff_t alloc_local, const ptrdiff_t *gc_below, const ptrdiff_t *gc_above,
    ptrdiff_t *local_ngc, ptrdiff_t *local_gc_start);
ptrdiff_t pfft_local_size_many_gc(
    int rnk_n, const ptrdiff_t *local_n, const ptrdiff_t *local_start, ptrdiff_t alloc_local,
    ptrdiff_t howmany, const ptrdiff_t *gc_below, const ptrdiff_t *gc_above,
    ptrdiff_t *local_ngc, ptrdiff_t *local_gc_start);
\end{lstlisting}
Works for complex and real arrays. R2c FFTs use the complex array decomposition for padded real values and the real array decomposition for non-padded real values.



\subsection{Plan real data ghost cells}
Local size fucntions are the same for complex and real data, see Section~\ref{sec:gc:local_size}.

\begin{lstlisting}
pfft_gcplan pfft_plan_rgc_3d(
    const ptrdiff_t *n, const ptrdiff_t *gc_below, const ptrdiff_t *gc_above,
    double *data, MPI_Comm comm_cart, unsigned gc_flags);
pfft_gcplan pfft_plan_rgc(
  int rnk_n, const ptrdiff_t *n, const ptrdiff_t *gc_below, const ptrdiff_t *gc_above,
  double *data, MPI_Comm comm_cart, unsigned gc_flags);
pfft_gcplan pfft_plan_many_rgc(
  int rnk_n, const ptrdiff_t *n, ptrdiff_t howmany, const ptrdiff_t *block,
  const ptrdiff_t *gc_below, const ptrdiff_t *gc_above, double *data, MPI_Comm comm_cart,
  unsigned gc_flags);
\end{lstlisting}

\subsection{Plan complex data ghost cells}
Local size fucntions are the same for complex and real data, see Section~\ref{sec:gc:local_size}.

\begin{lstlisting}
pfft_gcplan pfft_plan_cgc_3d(
    const ptrdiff_t *n, const ptrdiff_t *gc_below, const ptrdiff_t *gc_above,
    pfft_complex *data, MPI_Comm comm_cart, unsigned gc_flags);
pfft_gcplan pfft_plan_cgc(
    int rnk_n, const ptrdiff_t *n, const ptrdiff_t *gc_below, const ptrdiff_t *gc_above,
    pfft_complex *data, MPI_Comm comm_cart, unsigned gc_flags);
pfft_gcplan pfft_plan_many_cgc(
    int rnk_n, const ptrdiff_t *n, ptrdiff_t howmany, const ptrdiff_t *block,
    const ptrdiff_t *gc_below, const ptrdiff_t *gc_above, pfft_complex *data, MPI_Comm comm_cart,
    unsigned gc_flags);
\end{lstlisting}

\subsection{Timers for ghost cell communication}
\begin{lstlisting}
void pfft_reset_gctimers(
    pfft_gcplan ths);
pfft_gctimer pfft_get_gctimer_exg(
    const pfft_gcplan ths);
pfft_gctimer pfft_get_gctimer_red(
    const pfft_gcplan ths);
void pfft_print_average_gctimer(
    const pfft_gcplan ths, MPI_Comm comm);
void pfft_print_average_gctimer_adv(
    const pfft_gcplan ths, MPI_Comm comm);
void pfft_write_average_gctimer(
    const pfft_gcplan ths, const char *name, MPI_Comm comm);
void pfft_write_average_gctimer_adv(
    const pfft_gcplan ths, const char *name, MPI_Comm comm);

pfft_gctimer pfft_copy_gctimer(
    const pfft_gctimer orig);
void pfft_average_gctimer(
    pfft_gctimer ths);
pfft_gctimer pfft_add_gctimers(
    const pfft_gctimer sum1, const pfft_gctimer sum2);
pfft_gctimer pfft_reduce_max_gctimer(
    const pfft_gctimer ths, MPI_Comm comm);
void pfft_convert_gctimer2vec(
    const pfft_gctimer ths, double *times);
pfft_gctimer pfft_convert_vec2gctimer(
    const double *times);
void pfft_destroy_gctimer(
    pfft_gctimer ths);
\end{lstlisting}




\newpage
\section{Generalization to arbitrary dimensions of the FFT}
The following arrays are of length three for every functions ending with \code{_3d}
\begin{compactitem}
  \item \code{n,  local_n,  local_n_start}
  \item \code{ni, local_ni, local_i_start}
  \item \code{no, local_no, local_o_start}
\end{compactitem}
For all other functions the length of these arrays is given by the integer \code{rnk_n}.

\section{Generalization to arbitrary dimensions of the parallel decomposition}
The dimension of the parallel data decomposition is given by the communicator.
Simply use the functions described in Sect.~\ref{sec:create-comm} to create an communicator of arbitrary dimension
and the data decomposition will be adopted automatically.



\section{Using parallel plans}
The following functions directly correspond to their FFTW counterparts. Documentation is kept to a minimum with appropriate references to the FFTW documentation.

Before calling any PFFT function use
\begin{lstlisting}
void pfft_init(void);
\end{lstlisting}
to initialize the library. PFFT accumulates a small amount of persistent data during initialization and execution. Use
\begin{lstlisting}
void pfft_cleanup(void);
\end{lstlisting}
to release all of that. These two functions call \code{fftw_mpi_init} and \code{fftw_mpi_cleanup}; see also \cite{fftw-mpi-init}.
Especially, take care of the following fact analogously quoted from \cite{fftw-mpi-init}:
\begin{quote}
  After calling \code{pfft_cleanup}, all existing plans become undefined, and you should not attempt
  to execute or destroy them. You must call \code{pfft_init} again after \code{pfft_cleanup} if you want to resume using the PFFT routines.
\end{quote}
Execute any parallel FFT plan \code{fftplan} with
\begin{lstlisting}
void pfft_execute(pfft_plan fftplan);
\end{lstlisting}
and release the memory of a parallel FFT plan that is not needed anymore with
\begin{lstlisting}
void pfft_destroy_plan(pfft_plan fftplan);
\end{lstlisting}
Usage of these two functions is similar to their FFTW counterparts; see \cite{fftw-plans}.


\section{Memory allocation}
\begin{lstlisting}
void *pfft_malloc(size_t n);
void pfft_free(void *p);
\end{lstlisting}
These functions are a wrappers to \code{fftw_malloc}, \code{fftw_free} and are substitutes to the standard C functions \code{malloc}, \code{free}; see also \cite{fftw-malloc} for details.
Alternatively, use the shortcuts
\begin{lstlisting}
double *pfft_alloc_real(size_t n);
pfft_complex *pfft_alloc_complex(size_t n);
\end{lstlisting}
Hereby, the calls
\begin{lstlisting}
data_real = (double*) pfft_malloc(sizeof(double) * n);
data_cmpl = (pfft_complex*) pfft_malloc(sizeof(pfft_complex) * n);
\end{lstlisting}
are equivalent to
\begin{lstlisting}
data_real = pfft_malloc(n);
data_cmpl = pfft_malloc(n);
\end{lstlisting}

\section{Helper functions}
The following functions are useful tools but are not necessarily needed to perform parallel FFTs.

\section{Three-dimensional parallel FFT}
\subsection{Initializing inputs and checking outputs}\label{sec:init-data-3d-c2c}
To fill a complex array \code{data} with reproducible, complex values use one of the functions
\begin{lstlisting}
void pfft_init_input_complex_3d(
    const ptrdiff_t *n, const ptrdiff_t *local_n, const ptrdiff_t *local_n_start,
    pfft_complex *data);
void pfft_init_input_complex(
    int rnk_n, const ptrdiff_t *n, const ptrdiff_t *local_n, const ptrdiff_t *local_start,
    pfft_complex *data);
\end{lstlisting}
Hereby, the arrays \code{n}, \code{local_n} and \code{local_n_start} give the size of the FFT, the local array size and the local array offset;
see \cite{sec:par-data-decomp} for details on the parallel data decomposition.
The functions
\begin{lstlisting}
double pfft_check_output_complex_3d(
    const ptrdiff_t *n, const ptrdiff_t *local_n, const ptrdiff_t *local_n_start,
    const pfft_complex *data, MPI_Comm comm);
double pfft_check_output_complex(
    int rnk_n, const ptrdiff_t *n, const ptrdiff_t *local_n, const ptrdiff_t *local_start,
    const pfft_complex *data, MPI_Comm comm);
\end{lstlisting}
compute the $l_1$-norm between the elements of array \code{data} and values produced by \code{pfft_init_input_complex_3d}, \code{pfft_init_input_complex}.

Note, that these functions can be combined for a quick consistency check of the FFT.
Since a forward FFT followed by a backward FFT reproduces the inputs up to a scaling factor, the following code snippet should give a result equal to zero up to machine precision.
\begin{lstlisting}
/* Initialize input with random numbers */
pfft_init_input_complex_3d(n, local_ni, local_i_start,
    in);

/* execute parallel forward FFT */
pfft_execute(plan_forw);

/* execute parallel backward FFT */
pfft_execute(plan_back);

/* Scale data */
for(ptrdiff_t l=0; l < local_ni[0] * local_ni[1] * local_ni[2]; l++)
  in[l] /= (n[0]*n[1]*n[2]);

/* Print error of back transformed data */
err = pfft_check_output_complex_3d(n, local_ni, local_i_start, in, comm_cart_2d);
pfft_printf(comm_cart_2d, "Error after one forward and backward trafo of size n=(%td, %td, %td):\n", n[0], n[1], n[2]);
pfft_printf(comm_cart_2d, "maxerror = %6.2e;\n", err);
\end{lstlisting}




\subsection{Initializing real inputs and checking outputs}\label{sec:init-data-3d-r2r}
To fill a real array \code{data} with reproducible, complex values use one of the functions
\begin{lstlisting}
void pfft_init_input_real_3d(
    const ptrdiff_t *n, const ptrdiff_t *local_n, const ptrdiff_t *local_n_start,
    double *data);
void pfft_init_input_real(
    int rnk_n, const ptrdiff_t *n, const ptrdiff_t *local_n, const ptrdiff_t *local_start,
    double *data);
\end{lstlisting}
Hereby, the arrays \code{n}, \code{local_n} and \code{local_n_start} give the size of the FFT, the local array size and the local array offset;
see \cite{sec:par-data-decomp} for details on the parallel data decomposition.
The functions
\begin{lstlisting}
double pfft_check_output_real_3d(
    const ptrdiff_t *n, const ptrdiff_t *local_n, const ptrdiff_t *local_n_start,
    const pfft_complex *data, MPI_Comm comm);
double pfft_check_output_real(
    int rnk_n, const ptrdiff_t *n, const ptrdiff_t *local_n, const ptrdiff_t *local_start,
    const pfft_complex *data, MPI_Comm comm);
\end{lstlisting}
compute the $l_1$-norm between the elements of array \code{data} and values produced by \code{pfft_init_input_real_3d}, \code{pfft_init_input_real}.

Note, that both \code{pfft_init_input_real*} functions will set all array elements to zero were \code{local_n + local_n_start >= n}.
In addition, both \code{pfft_check_output_real*} function will ignore all the errors resulting from these elements.
Therefore, it is safe to use all these functions for a consistency check of a r2c transform followed by a c2r transform since all padding elements will be ignored.


\subsection{Initializing r2c/c2r inputs and checking outputs}\label{sec:init-data-3d-r2c}
The real inputs of a r2c transform can be initialized with the functions decribed in Section~\ref{sec:init-data-3d-r2r}.
However, generating suitable inputs for a c2r transform requires more caution.
In order to get real valued results of a DFT the complex input coefficients need to satisfy an radial Hermitian symmetry, i.e., $X[\mathbf k] = {X^*[-\mathbf k]}$.
We use the following trick to generate the complex input values for c2r transforms.
Assume any $\mathbf N$-periodic complex valued function $f$. It can be easily shown that the values
$X[\mathbf k] := \frac{1}{2}\left(f(\mathbf k)+f^*(-\mathbf k)\right)$ satisfy the radial Hermitian symmetry. 

To fill a complex array \code{data} with reproducible, complex values that fulfill the radial Hermitian symmetry use one of the functions
\begin{lstlisting}
void pfft_init_input_complex_hermitian_3d(
    const ptrdiff_t *n, const ptrdiff_t *local_n, const ptrdiff_t *local_n_start,
    double *data);
void pfft_init_input_complex_hermitian(
    int rnk_n, const ptrdiff_t *n, const ptrdiff_t *local_n, const ptrdiff_t *local_start,
    double *data);
\end{lstlisting}
Hereby, the arrays \code{n}, \code{local_n} and \code{local_n_start} give the size of the FFT, the local array size and the local array offset;
see \cite{sec:par-data-decomp} for details on the parallel data decomposition.
The functions
\begin{lstlisting}
double pfft_check_output_complex_hermitian_3d(
    const ptrdiff_t *n, const ptrdiff_t *local_n, const ptrdiff_t *local_n_start,
    const pfft_complex *data, MPI_Comm comm);
double pfft_check_output_complex_hermitian(
    int rnk_n, const ptrdiff_t *n, const ptrdiff_t *local_n, const ptrdiff_t *local_start,
    const pfft_complex *data, MPI_Comm comm);
\end{lstlisting}
compute the $l_1$-norm between the elements of array \code{data} and values produced by \code{pfft_init_input_complex_hermitian_3d}, \code{pfft_init_input_complex_hermitian}.

Note, that these functions can also be used in order to generate complex inputs with radial Hermitian symmetry for ordinary c2c transforms. 
Of course the results of such a c2c DFT will have all imaginary parts equal to zero up to machine precision.










\section{Generalization to arbitrary dimensions}

\begin{lstlisting}
void pfft_init_input_real_3d(
    const ptrdiff_t *n, const ptrdiff_t *local_n, const ptrdiff_t *local_n_start,
    double *data);
void pfft_init_input_real(
    int rnk_n, const ptrdiff_t *n, const ptrdiff_t *local_n, const ptrdiff_t *local_start,
    double *data);
\end{lstlisting}

\begin{lstlisting}
double pfft_check_output_real_3d(
    const ptrdiff_t *n, const ptrdiff_t *local_n, const ptrdiff_t *local_n_start,
    const double *data, MPI_Comm comm);
double pfft_check_output_real(
    int rnk_n, const ptrdiff_t *n, const ptrdiff_t *local_n, const ptrdiff_t *local_start,
    const double *data, MPI_Comm comm);
\end{lstlisting}


\subsection{Planning parallel FFTs}

\subsubsection{Basic interface}
The \code{_3d} interface is the simplest interface layer. It is suitable for the planning of three-dimensional FFTs.
\begin{lstlisting}
ptrdiff_t pfft_local_size_dft_3d(
    const ptrdiff_t *n, MPI_Comm comm_cart, unsigned pfft_flags,
    ptrdiff_t *local_ni, ptrdiff_t *local_i_start,
    ptrdiff_t *local_no, ptrdiff_t *local_o_start);
ptrdiff_t pfft_local_size_dft_r2c_3d(
    const ptrdiff_t *n, MPI_Comm comm_cart, unsigned pfft_flags,
    ptrdiff_t *local_ni, ptrdiff_t *local_i_start,
    ptrdiff_t *local_no, ptrdiff_t *local_o_start);
ptrdiff_t pfft_local_size_dft_c2r_3d(
    const ptrdiff_t *n, MPI_Comm comm_cart, unsigned pfft_flags,
    ptrdiff_t *local_ni, ptrdiff_t *local_i_start,
    ptrdiff_t *local_no, ptrdiff_t *local_o_start);
ptrdiff_t pfft_local_size_r2r_3d(
    const ptrdiff_t *n, MPI_Comm comm_cart, unsigned pfft_flags,
    ptrdiff_t *local_ni, ptrdiff_t *local_i_start, ptrdiff_t *local_no, ptrdiff_t *local_o_start);
\end{lstlisting}
\begin{lstlisting}
pfft_plan pfft_plan_dft_3d(
    const ptrdiff_t *n, pfft_complex *in, pfft_complex *out,
    MPI_Comm comm_cart, int sign, unsigned pfft_flags);
pfft_plan pfft_plan_dft_r2c_3d(
    const ptrdiff_t *n, double *in, pfft_complex *out,
    MPI_Comm comm_cart, int sign, unsigned pfft_flags);
pfft_plan pfft_plan_dft_c2r_3d(
    const ptrdiff_t *n, pfft_complex *in, double *out,
    MPI_Comm comm_cart, int sign, unsigned pfft_flags);
pfft_plan pfft_plan_r2r_3d(
    const ptrdiff_t *n, double *in, double *out,
    MPI_Comm comm_cart, const pfft_r2r_kind *kinds,
    unsigned pfft_flags);
\end{lstlisting}

The basic interface generalizes the \code{_3d} interface to FFTs of arbitrary dimension \code{rnk}.
\begin{lstlisting}
ptrdiff_t pfft_local_size_dft(
    int rnk, const ptrdiff_t *n,
    MPI_Comm comm_cart, unsigned pfft_flags,
    ptrdiff_t *local_ni, ptrdiff_t *local_i_start,
    ptrdiff_t *local_no, ptrdiff_t *local_o_start);
ptrdiff_t pfft_local_size_dft_r2c(
    int rnk, const ptrdiff_t *n,
    MPI_Comm comm_cart, unsigned pfft_flags,
    ptrdiff_t *local_ni, ptrdiff_t *local_i_start,
    ptrdiff_t *local_no, ptrdiff_t *local_o_start);
ptrdiff_t pfft_local_size_dft_c2r(
    int rnk, const ptrdiff_t *n,
    MPI_Comm comm_cart, unsigned pfft_flags,
    ptrdiff_t *local_ni, ptrdiff_t *local_i_start,
    ptrdiff_t *local_no, ptrdiff_t *local_o_start);
ptrdiff_t pfft_local_size_r2r(
    int rnk_n, const ptrdiff_t *n,
    MPI_Comm comm_cart, unsigned pfft_flags,
    ptrdiff_t *local_ni, ptrdiff_t *local_i_start,
    ptrdiff_t *local_no, ptrdiff_t *local_o_start);
\end{lstlisting}
\begin{lstlisting}
pfft_plan pfft_plan_dft(
    int rnk, const ptrdiff_t *n, pfft_complex *in, pfft_complex *out, MPI_Comm comm_cart,
    int sign, unsigned pfft_flags);
pfft_plan pfft_plan_dft_r2c(
    int rnk_n, const ptrdiff_t *n, double *in, pfft_complex *out, MPI_Comm comm_cart,
    int sign, unsigned pfft_flags);
pfft_plan pfft_plan_dft_c2r(
    int rnk_n, const ptrdiff_t *n, pfft_complex *in, double *out, MPI_Comm comm_cart,
    int sign, unsigned pfft_flags);
pfft_plan pfft_plan_r2r(
    int rnk_n, const ptrdiff_t *n, double *in, double *out, MPI_Comm comm_cart,
    const pfft_r2r_kind *kinds, unsigned pfft_flags);
\end{lstlisting}
Therefore, \code{n} and \code{kinds} become arrays of length \code{rnk_n}.

\subsubsection{Advanced interface}
The advanced interface introduces the arrays \code{ni} and \code{no} of length \code{rnk_n}
that give the pruned FFT input and output size.
Furthermore, the arrays \code{iblock} and \code{oblock} of length \code{rnk_pm} (\code{rnk_pm} being the dimension of the process mesh)
serve to adjust the block size of the input and output block decomposition.
\begin{lstlisting}
ptrdiff_t pfft_local_size_many_dft(
    int rnk, const ptrdiff_t *n, const ptrdiff_t *ni, const ptrdiff_t *no,
    ptrdiff_t howmany, const ptrdiff_t *iblock, const ptrdiff_t *oblock,
    MPI_Comm comm_cart, unsigned pfft_flags,
    ptrdiff_t *local_ni, ptrdiff_t *local_i_start,
    ptrdiff_t *local_no, ptrdiff_t *local_o_start);
ptrdiff_t pfft_local_size_many_dft_r2c(
    int rnk_n, const ptrdiff_t *n, const ptrdiff_t *ni, const ptrdiff_t *no,
    ptrdiff_t howmany, const ptrdiff_t *iblock, const ptrdiff_t *oblock,
    MPI_Comm comm_cart, unsigned pfft_flags,
    ptrdiff_t *local_ni, ptrdiff_t *local_i_start,
    ptrdiff_t *local_no, ptrdiff_t *local_o_start);
ptrdiff_t pfft_local_size_many_dft_c2r(
    int rnk_n, const ptrdiff_t *n, const ptrdiff_t *ni, const ptrdiff_t *no,
    ptrdiff_t howmany, const ptrdiff_t *iblock, const ptrdiff_t *oblock,
    MPI_Comm comm_cart, unsigned pfft_flags,
    ptrdiff_t *local_ni, ptrdiff_t *local_i_start,
    ptrdiff_t *local_no, ptrdiff_t *local_o_start);
ptrdiff_t pfft_local_size_many_r2r(
    int rnk_n, const ptrdiff_t *n, const ptrdiff_t *ni, const ptrdiff_t *no,
    ptrdiff_t howmany, const ptrdiff_t *iblock, const ptrdiff_t *oblock,
    MPI_Comm comm_cart, unsigned pfft_flags,
    ptrdiff_t *local_ni, ptrdiff_t *local_i_start,
    ptrdiff_t *local_no, ptrdiff_t *local_o_start);
\end{lstlisting}
\begin{lstlisting}
pfft_plan pfft_plan_many_dft(
    int rnk, const ptrdiff_t *n, const ptrdiff_t *ni, const ptrdiff_t *no,
    ptrdiff_t howmany, const ptrdiff_t *iblock, const ptrdiff_t *oblock,
    pfft_complex *in, pfft_complex *out, MPI_Comm comm_cart,
    int sign, unsigned pfft_flags);
pfft_plan pfft_plan_many_dft_r2c(
    int rnk_n, const ptrdiff_t *n, const ptrdiff_t *ni, const ptrdiff_t *no,
    ptrdiff_t howmany, const ptrdiff_t *iblock, const ptrdiff_t *oblock,
    double *in, pfft_complex *out, MPI_Comm comm_cart,
    int sign, unsigned pfft_flags);
pfft_plan pfft_plan_many_dft_c2r(
    int rnk_n, const ptrdiff_t *n, const ptrdiff_t *ni, const ptrdiff_t *no,
    ptrdiff_t howmany, const ptrdiff_t *iblock, const ptrdiff_t *oblock,
    pfft_complex *in, double *out, MPI_Comm comm_cart,
    int sign, unsigned pfft_flags);
pfft_plan pfft_plan_many_r2r(
    int rnk_n, const ptrdiff_t *n, const ptrdiff_t *ni, const ptrdiff_t *no,
    ptrdiff_t howmany, const ptrdiff_t *iblock, const ptrdiff_t *oblock,
    double *in, double *out, MPI_Comm comm_cart,
    const pfft_r2r_kind *kinds, unsigned pfft_flags);
\end{lstlisting}

\subsubsection{New-array execute}
Similar to FFTW we have
\begin{lstlisting}[escapechar=§]
void pfft_execute(const pfft_plan plan);
void pfft_execute_dft(const pfft_plan plan, pfft_complex *in, pfft_complex *out);
void pfft_execute_dft_r2c(const pfft_plan plan, double *in, pfft_complex *out);
void pfft_execute_dft_c2r(const pfft_plan plan, pfft_complex *in, double *out);
void pfft_execute_r2r(const pfft_plan plan, double *in, double *out);
\end{lstlisting}



\subsubsection{Preliminary: Skip selected serial transformations}
The \code{_skipped} interface is based on the \code{many} interface but adds the possibility to skip selected serial FFTs.
\begin{lstlisting}[escapechar=§]
pfft_plan pfft_plan_many_dft_skipped(
    int rnk, const ptrdiff_t *n,
    const ptrdiff_t *ni, const ptrdiff_t *no,
    ptrdiff_t howmany,
    const ptrdiff_t *iblock, const ptrdiff_t *oblock,
    (red@const int *skip_trafos@*),
    pfft_complex *in, pfft_complex *out, MPI_Comm comm_cart,
    int sign, unsigned pfft_flags);
\end{lstlisting}
Hereby, \code{skip_trafos} is an integer array of length \code{rnk_pm+1} (\code{rnk_pm} being the dimension of the communicator \code{comm_cart}).
For \code{t=0,...,rnk_pm} set \code{skip_trafos[t]=1} if the \code{t}-th serial transformation should be computed, otherwise set \code{skip_trafos[t]=0}.
Note that the local transpositions are always performed.

\section{Useful tools}
\subsection{Operations on arrays of type \code{ptrdiff_t}}
The following routines are shortcuts for the elementwise manipulation of \code{ptrdiff_t} arrays.
In the following, all arrays  \code{vec}, \code{vec1}, and \code{vec2} are of length \code{d} and type \code{ptrdiff_t}.
\begin{lstlisting}
ptrdiff_t pfft_prod_ptrdiff_t(
    int d, const ptrdiff_t *vec);
\end{lstlisting}
Returns the product over all elements of \code{vec}.
\begin{lstlisting}
ptrdiff_t pfft_sum_ptrdiff_t(
    int d, const ptrdiff_t *vec);
\end{lstlisting}
Returns the sum over all elements of \code{vec}.
\begin{lstlisting}
int pfft_equal_ptrdiff_t(
    int d, const ptrdiff_t *vec1, const ptrdiff_t *vec2);
\end{lstlisting}
Returns 1 if both arrays have equal entries, 0 otherwise.
\begin{lstlisting}
void pfft_vcopy_ptrdiff_t(
    int d, const ptrdiff_t *vec1,
    ptrdiff_t *vec2);
\end{lstlisting}
Copies the elements of \code{vec1} into \code{vec2}.
\begin{lstlisting}
void pfft_vadd_ptrdiff_t(
    int d, const ptrdiff_t *vec1, const ptrdiff_t *vec2,
    ptrdiff_t *sum);
\end{lstlisting}
Fills \code{sum} with the componentwise sum of \code{vec1} and \code{vec2}.
\begin{lstlisting}
void pfft_vsub_ptrdiff_t(
    int d, const ptrdiff_t *vec1, const ptrdiff_t *vec2,
    ptrdiff_t *sum);
\end{lstlisting}
Fills \code{sum} with the componentwise difference of \code{vec1} and \code{vec2}.

\subsection{Print three-dimensional arrays in parallel}
Use the following routine to print the elements of a block decomposed three-dimensional array \code{data} in a nicely formatted way.
\begin{lstlisting}
void pfft_apr_complex_3d(
    const pfft_complex *data,
    const ptrdiff_t *local_n, const ptrdiff_t *local_start,
    const char *name, MPI_Comm comm);
\end{lstlisting}
However, this makes only sense for arrays of moderate size.
The block decomposition is given by \code{local_n}, \code{local_n_start} as described in see Section~\ref{sec:par-data-decomp}.
Furthermore, some arbitrary string \code{name} can be added at the beginning of each output - typically this will be the name of the array.
Communicator \code{comm} must be suitable to the block decomposition and is used to synchronize the outputs over all processes.

A generalization for the case where the dimensions of the local arrays are permuted is given by
\begin{lstlisting}
void pfft_apr_complex_permuted_3d(
    const pfft_complex *data,
    const ptrdiff_t *local_n, const ptrdiff_t *local_start,
    int perm0, int perm1, int perm2,
    const char *name, MPI_Comm comm);
\end{lstlisting}
Hereby, \code{perm0}, \code{perm1}, and \code{perm2} give the array's permutation of dimension.

\subsection{Reading command line arguments}
The following function offers a simple way to read command line arguments into an array \code{parameter}.
\begin{lstlisting}
void pfft_get_args(
    int argc, char **argv, const char *name,
    int neededArgs, unsigned type,
    void *parameter);
\end{lstlisting}
Hereby, \code{argc} and \code{argv} are the standard argument of the \code{main} routine.
Furthermore, \code{name}, \code{neededAgrs}, and \code{type} give the name, number of entries and the type of the command line argument.
Supported types are \code{PFFT_INT}, \code{PFFT_PTRDIFF_T}, \code{PFFT_FLOAT}, \code{PFFT_DOUBLE}, and \code{PFFT_UNSIGNED},
which denote the standard C type that is used for typecasting.
The array \code{parameter} must be of sufficient size to hold \code{neededArgs} elements of the given data type.

For example, a program \code{test.c} containing the following code snippet
\begin{lstlisting}
double x;
pfft_get_args(argc, argv, "-pfft_x", 1, PFFT_DOUBLE, &x);
int np[2];
pfft_get_args(argc, argv, "-pfft_np", 2, PFFT_INT, np);
ptrdiff_t n[3];
pfft_get_args(argc, argv, "-pfft_n", 3, PFFT_PTRDIFF_T, n);
\end{lstlisting}
and run with
\begin{lstlisting}
./test -pfft_x 3.1 -pfft_np 2 3 -pfft_n 8 16 32
\end{lstlisting}
will read \code{x=3.1}, \code{np[2] = \{2,3\}}, and \code{n[3]=\{8,16,32\}}.
The argument order of appearance in the command line is irrelevant.
Take care of the address operator \code{&} in front of \code{x} in the second line!

\subsection{Parallel substitutes for \code{vprintf}, \code{fprintf}, and \code{printf}}\label{sec:fprintf}
The following functions are similar to the standard C function \code{vfprintf}, \code{fprintf} and \code{printf} with the exception,
that all messages are restricted to the process of rank $0$ within the given communicator \code{comm}.
\begin{lstlisting}
void pfft_vfprintf(
    MPI_Comm comm, FILE *stream, const char *format, va_list ap);
void pfft_fprintf(
    MPI_Comm comm, FILE *stream, const char *format, ...);
void pfft_printf(
    MPI_Comm comm, const char *format, ...);
\end{lstlisting}

\section{Generating periodic Cartesian communicators}\label{sec:create-comm}
\begin{lstlisting}
int pfft_create_procmesh_2d(
    MPI_Comm comm, int np0, int np1,
    MPI_Comm *comm_cart_2d);
\end{lstlisting}
This function uses the process group of a given communicator \code{comm} to create and allocate a
two-dimensional, periodic, Cartesian communicator \code{comm_cart_2d} of size \code{np0*np1}.
The memory of the generated communicator can be released with \code{MPI_Comm_free}.
\begin{lstlisting}
int pfft_create_procmesh(
    int rnk, MPI_Comm comm, const int *np,
    MPI_Comm *comm_cart);
\end{lstlisting}
Use the process group of a given communicator \code{comm} to create and allocate a
\code{rnk}-dimensional, periodic, Cartesian communicator of size \code{np[0]*np[1]*...*np[rnk-1]}.
Hereby, \code{np} is an array of size \code{rnk}.
The memory of the generated communicator should be released with \code{MPI_Comm_free} after usage.

\subsection{Parallel data decomposition}\label{sec:par-data-decomp}



\section{PFFT timers}
PFFT offers an easy way to perform run time measurements and print/write the results.

\subsection{Basis run time measurements}
PFFT-plans automatically accumulate the local run times of every call to \code{pfft_execute}.
For most applications it is sufficient to print run time of a plan \code{ths} averaged over all runs with
\begin{lstlisting}
void pfft_print_average_timer(
    const pfft_plan ths, MPI_Comm comm);
\end{lstlisting}
Note, that for each timer the maximum over all processes is computed on the first process (rank 0) of communicator \code{comm},
i.e., a call to \code{MPI_Reduce} is performed and the output is only printed on this process.
The following function works in the same way but prints more verbose output
\begin{lstlisting}
void pfft_print_average_timer_adv(
    const pfft_plan ths, MPI_Comm comm);
\end{lstlisting}

To write the averaged run time of plan \code{ths} into a file called \code{name} use
\begin{lstlisting}
void pfft_write_average_timer(
    const pfft_plan ths, const char *name, MPI_Comm comm);
void pfft_write_average_timer_adv(
    const pfft_plan ths, const char *name, MPI_Comm comm);
\end{lstlisting}
The output is only written on the first rank of communicator \code{comm}.

Discard all the recorded run times with
\begin{lstlisting}
void pfft_reset_timer(
    pfft_plan ths);
\end{lstlisting}
This function is called per default when a new plan is created.

\subsection{Advanced timer manipulation}
In order to access the run times directly a new typedef \code{pfft_timer} is introduced.
The following function returns a copy of the timer corresponding to plan \code{ths}
\begin{lstlisting}
pfft_timer pfft_get_timer(
    const pfft_plan ths);
\end{lstlisting}





Note that all functions with a \code{pfft_timer} return value automatically allocate memory

Note that the user must Whenever the return value of a function is a \code{pfft_timer}, the user is responsible 


must release it's memory with
\begin{lstlisting}
void pfft_destroy_timer(
    pfft_timer ths);
\end{lstlisting}
as soon as the timer is not needed anymore. 

In the following we introduce some routines to perform basic operations on timers.
Create a copy of a PFFT-timer \code{orig} with
\begin{lstlisting}
pfft_timer pfft_copy_timer(
    const pfft_timer orig);
\end{lstlisting}
Compute the average, local time over all runs of a timer \code{ths} with
\begin{lstlisting}
void pfft_average_timer(
    pfft_timer ths);
\end{lstlisting}
Create a new timer that contains the sum of two timers \code{sum1} and \code{sum2} with
\begin{lstlisting}
pfft_timer pfft_add_timers(
    const pfft_timer sum1, const pfft_timer sum2);
\end{lstlisting}
Create a timer that contains the maximum times of all the timers \code{ths} from all processes belonging to communicator \code{comm} with
\begin{lstlisting}
pfft_timer pfft_reduce_max_timer(
    const pfft_timer ths, MPI_Comm comm);
\end{lstlisting}
Since this function calls \code{MPI_Reduce}, only the first process (rank 0) of \code{comm} will get the desired data while all
the other processes have timers with undefined values.

Note, that you can not access the elements of a timer directly, since it is only a pointer to a \code{struct}.
However, PFFT offers a routine that creates an array and copies all the entries of the timer into it
\begin{lstlisting}
double* pfft_convert_timer2vec(
    const pfft_timer ths);
\end{lstlisting}
The entries of the returned array are ordered as follows:
\begin{compactitem}
  \item dimension of the process mesh \code{rnk_pm}
  \item number of serial trafos \code{rnk_trafo}
  \item number of global remaps \code{rnk_remap}
  \item number of \code{pfft_execute} runs \code{iter}
  \item local run time of all runs
  \item \code{rnk_n} local times of the serial trafos
  \item \code{rnk_remap} local times of the global remaps
  \item 2 times of the global remaps that are only necessary for three-dimensional FFTs on three-dimensional process meshes
\end{compactitem}

The complementary function
\begin{lstlisting}
pfft_timer pfft_convert_vec2timer(
    const double *times);
\end{lstlisting}
creates a timer and fills it's entries with the data from array \code{times}.


\section{Ghost cells}

\subsection{Determine the parallel ghost cell layout}

Similarly to the

In general, local ghost cell layout is characterized by the local array size $local_ngc[rnk_n]$ and
the local offset \code{local_gc_start[rnk_n]}. Every process gets the data

A process owns the local array data
\code{data[k[0],k[1],k[2]]} with \code{local_start[t] <= k[t] < local_start[t] + local_n[t]},
add the ghost cells below and above, such that
\code{data[k[0],k[1],k[2]]} with \code{gc_below + local_start[t] <= k[t] < local_start[t] + local_n[t]},

Hereby, \code{local_gc_start[t] = local_start[t] - gc_below[t]} and \code{local_ngc[t] = local_n[t] + local_gc_below[t] + local_gc_above[t]}.

\code{data[k[0],k[1],k[2]]} with \code{local_gc_start[t] <= k[t] < local_gc_start[t] + local_ngc[t]}

The output arrays \code{local_ngc} and \code{local_gc_start} give the size and the offset of the local array plus ghost cells, i.e.,
every process owns the data \code{data[k[0],k[1],k[2]]} with
\begin{lstlisting}
local_gc_start[t] <= k[t] < local_gc_start[t] + local_ngc[t]
\end{lstlisting}
for \code{t=0,...,rnk-1}.
Hereby, the index \code{k} wraps periodically modulo \code{n} such that \code{0 <= k[t] < n[t]}.

\begin{lstlisting}
ptrdiff_t pfft_local_size_gc_3d(
    const ptrdiff_t *local_n, const ptrdiff_t *local_start,
    ptrdiff_t alloc_local, const ptrdiff_t *gc_below, const ptrdiff_t *gc_above,
    ptrdiff_t *local_ngc, ptrdiff_t *local_gc_start);
\end{lstlisting}
For a given local array of size \code{local_n} with offset \code{local_n_start} compute the
local array plus ghost cell size \code{local_ngc} and offset \code{local_gc_start}.
The number of ghost cells below and above the local array are given by
Return value is the necessary memory to store the array plus ghost cells.
Hereby, \code{alloc_local} should be the return value of an appropriate
call of a \code{pff_local_size} function, see \ref{sec:local-size}.

\begin{lstlisting}
ptrdiff_t pfft_local_size_gc(
    int rnk_n, const ptrdiff_t *local_n, const ptrdiff_t *local_start,
    ptrdiff_t alloc_local, const ptrdiff_t *gc_below, const ptrdiff_t *gc_above,
    ptrdiff_t *local_ngc, ptrdiff_t *local_gc_start);
\end{lstlisting}
Generalize the three-dimensional interface to \code{rnk} dimensions.
\begin{lstlisting}
ptrdiff_t pfft_local_size_many_gc(
    int rnk_n, const ptrdiff_t *local_n, const ptrdiff_t *local_start, ptrdiff_t alloc_local,
    ptrdiff_t howmany, const ptrdiff_t *gc_below, const ptrdiff_t *gc_above,
    ptrdiff_t *local_ngc, ptrdiff_t *local_gc_start);
\end{lstlisting}

\begin{lstlisting}
pfft_gcplan pfft_plan_rgc_3d(
    const ptrdiff_t *n, const ptrdiff_t *gc_below, const ptrdiff_t *gc_above,
    double *data, MPI_Comm comm_cart, unsigned gc_flags);
pfft_gcplan pfft_plan_cgc_3d(
    const ptrdiff_t *n, const ptrdiff_t *gc_below, const ptrdiff_t *gc_above,
    pfft_complex *data, MPI_Comm comm_cart, unsigned gc_flags);
pfft_gcplan pfft_plan_rgc(
    int rnk_n, const ptrdiff_t *n, const ptrdiff_t *gc_below, const ptrdiff_t *gc_above,
    double *data, MPI_Comm comm_cart, unsigned gc_flags);
pfft_gcplan pfft_plan_cgc(
    int rnk_n, const ptrdiff_t *n, const ptrdiff_t *gc_below, const ptrdiff_t *gc_above,
    pfft_complex *data, MPI_Comm comm_cart, unsigned gc_flags);
pfft_gcplan pfft_plan_many_rgc(
    int rnk_n, const ptrdiff_t *n, ptrdiff_t howmany, const ptrdiff_t *block,
    const ptrdiff_t *gc_below, const ptrdiff_t *gc_above, double *data, MPI_Comm comm_cart,
    unsigned gc_flags);
pfft_gcplan pfft_plan_many_cgc(
    int rnk_n, const ptrdiff_t *n, ptrdiff_t howmany, const ptrdiff_t *block,
    const ptrdiff_t *gc_below, const ptrdiff_t *gc_above, pfft_complex *data, MPI_Comm comm_cart,
    unsigned gc_flags);
\end{lstlisting}

\begin{lstlisting}
void pfft_exchange(
    pfft_gcplan ths);
void pfft_reduce(
    pfft_gcplan ths);
void pfft_destroy_gcplan(
    pfft_gcplan ths);
\end{lstlisting}

\subsection{Ghost cell timers}
\begin{lstlisting}
void pfft_reset_gctimers(
    pfft_gcplan ths);
pfft_gctimer pfft_get_gctimer_exg(
    const pfft_gcplan ths);
pfft_gctimer pfft_get_gctimer_red(
    const pfft_gcplan ths);
void pfft_print_average_gctimer(
    const pfft_gcplan ths, MPI_Comm comm);
void pfft_print_average_gctimer_adv(
    const pfft_gcplan ths, MPI_Comm comm);
void pfft_write_average_gctimer(
    const pfft_gcplan ths, const char *name, MPI_Comm comm);
void pfft_write_average_gctimer_adv(
    const pfft_gcplan ths, const char *name, MPI_Comm comm);
\end{lstlisting}

\begin{lstlisting}
pfft_gctimer pfft_copy_gctimer(
    const pfft_gctimer orig);
void pfft_average_gctimer(
    pfft_gctimer ths);
pfft_gctimer pfft_add_gctimers(
    const pfft_gctimer sum1, const pfft_gctimer sum2);
pfft_gctimer pfft_reduce_max_gctimer(
    const pfft_gctimer ths, MPI_Comm comm);
void pfft_convert_gctimer2vec(
    const pfft_gctimer ths, double *times);
pfft_gctimer pfft_convert_vec2gctimer(
    const double *times);
void pfft_destroy_gctimer(
    pfft_gctimer ths);
\end{lstlisting}

\begin{lstlisting}
typedef struct {
  int iter;
  double whole;
  double pad_zeros;
  double exchange;
} PX(gctimer_s);
typedef PX(gctimer_s) *PX(gctimer);
\end{lstlisting}



% use this line in gvim to substitute PX(func_name_01) by pfft_func_name_01
%s/PX(\([a-z_0-9]*\))/pfft_\1/gc
