%%%%%%%%%%%%%%%%%%%%%%%%%%%%%%%%%%%%%%%%%%%%%%%%%%%%%%%%%%%%%%%%%%%%%%%%%%%%%%%
\chapter{Developers Guide}\label{chap:develop}
%%%%%%%%%%%%%%%%%%%%%%%%%%%%%%%%%%%%%%%%%%%%%%%%%%%%%%%%%%%%%%%%%%%%%%%%%%%%%%%

\section{PFFT Interface Layer Specification}
\begin{compactitem}
  \myitem start with internal interface and simplify it into basic interface
\end{compactitem}


\emph{Gcell Internal Interface}
\begin{compactitem}
  \item[+] \verb++
  \item[+] \verb++
  \item[+] \verb++
\end{compactitem}




\emph{Basic Interface}
\begin{compactitem}
  \item[+] \verb+const INT *n+
  \item[+] \verb+C *in+
  \item[+] \verb+C *out+
  \item[+] \verb+MPI\_Comm comm\_cart\_2d+
  \item[+] \verb+int sign+
  \item[+] \red{unsigned pfft\_flags}
  \item[+] \verb+unsigned fftw\_flags+
\end{compactitem}

\emph{3D Basic Interface}
\begin{compactitem}
  \item[-] \verb+const INT *n+
  \item[+] \verb+INT n0, n1, n2+
\end{compactitem}

\emph{3D Basic Interface Transposed}
\begin{compactitem}
  \item[-] \verb+const INT *n+
  \item[+] \verb+INT n0, n1, n2+
  \item[+] \red{unsigned pfft\_flags}
\end{compactitem}

\emph{Advanced Interface}
\begin{compactitem}
  \item[+] \verb+const INT *ni+
  \item[+] \verb+const INT *no+
  \item[+] \verb+INT howmany+
  \item[+] \verb+const INT *block_col+
  \item[+] \verb+const INT *block_row+
\end{compactitem}

\emph{Guru Interface}
\begin{compactitem}
  \item[-] \red{unsigned pfft\_flags}
  \item[+] \verb+pfft\_profile *prf+
  \item[+] \verb+pfft\_permutation *perm+
\end{compactitem}

\emph{Internal Interface}
\begin{compactitem}
  \item[$-$] \verb+MPI_Comm comm_cart_2d+
  \item[+] \red{MPI\_Comm comm\_col}
  \item[+] \red{MPI\_Comm comm\_row}
  \item[+] \verb+pfft_truncation *trc+
  \item[+] \verb+user_blocksize+ $\Rightarrow$ \verb+block_size_internal+
  \item[+] \verb+howmany+ $\Rightarrow$ \verb+tuplesize+
  \item[+] \verb+fftw_flags+ $\Rightarrow$ \verb+fftw_planing_flags+
\end{compactitem}


\newcommand{\redlst}[1]{\lstinline+#1+}

\newpage
\subsection{PFFT Interface - Headers}
\emph{New PFFT Interface: local size (Internal, Guru, Adv, Basic, 3d, 2d)}
\begin{lstlisting}
INT PX(local_size_internal)(
    int (*@\color{red}rnk\_nc@*), const INT *n, const INT *ni, const INT *no,
    INT (*@\color{red}tuple\_size@*), const INT *iblock, const INT *oblock,
    int (*@\color{red}rnk\_pm@*), MPI_Comm (*@\color{red}*comms\_pm@*),
    INT *local_ni, INT *local_i_start,
    INT *local_no, INT *local_o_start);
// blocks are internal blocks (NULLs are substituted by default values)
// use MPI_Cart_split multiple times to get comms_1d
// canonical PFFT:
// one more array dimension, than procmesh dimension
// tuple_size includes howmany and the last dimensions

INT PX(local_size_guru)(
    int rnk, const INT *n, const INT *ni, const INT *no,
    const (*@\red{PX(profile) *prf}@*), const (*@\color{red}PX(permutation) *perm@*),
    INT (*@\color{green}howmany@*), const INT *iblock, const INT *oblock,
    MPI_Comm (*@\color{green}comm\_cart@*),
    INT *local_ni, INT *local_i_start,
    INT *local_no, INT *local_o_start,
    PX(permutation) (*@\color{orange}*perm\_out)@*);
/* !!! prf depends on perm !!! */
/* !!! oblock depends on perm, prf, no !!! */

INT PX(local_size_many_transposed)(
    int rnk, const INT *n, const INT (*@\color{red}*ni@*), const INT (*@\color{red}*no@*),
    INT (*@\color{red}howmany@*), const INT (*@\color{red}*iblock@*), const INT (*@\color{red}*oblock@*),
    MPI_Comm comm_cart
    INT *local_ni, INT *local_i_start,
    INT *local_no, INT *local_o_start);
INT PX(local_size_many)(
    int rnk, const INT *n, const INT (*@\color{red}*ni@*), const INT (*@\color{red}*no@*),
    INT (*@\color{red}howmany@*), const INT (*@\color{red}*iblock@*), const INT (*@\color{red}*oblock@*),
    MPI_Comm comm_cart
    INT *local_ni, INT *local_i_start,
    INT (*@\color{red}*local\_no@*), INT (*@\color{red}*local\_o\_start@*));

INT PX(local_size_transposed)(
    int (*@\color{red}rnk@*), const INT (*@\color{red}*n@*), MPI_Comm comm_cart,
    INT (*@\color{red}*local\_ni@*), INT (*@\color{red}*local\_i\_start@*)
    INT (*@\color{red}*local\_no@*), INT (*@\color{red}*local\_o\_start@*));
INT PX(local_size)(
    int (*@\color{red}rnk@*), const INT (*@\color{red}*n@*), MPI_Comm comm_cart,
    INT (*@\color{red}*local\_ni@*), INT (*@\color{red}*local\_i\_start@*));

INT PX(local_size_3d_transposed)(
    INT (*@\color{green}n0@*), INT (*@\color{green}n1@*), (*@\color{orange}INT n2@*), MPI_Comm (*@\color{red}comm\_cart@*),
    INT (*@\color{green}*local\_ni0@*), INT (*@\color{green}*local\_i0\_start@*),
    INT (*@\color{orange}*local\_ni1@*), INT (*@\color{orange}*local\_i1\_start@*),
    INT (*@\color{green}*local\_no1@*), INT (*@\color{green}*local\_o1\_start@*),
    INT (*@\color{orange}*local\_no2@*), INT (*@\color{orange}*local\_o2\_start@*));
INT PX(local_size_3d)(
    INT (*@\color{green}n0@*), INT (*@\color{green}n1@*), (*@\color{orange}INT n2@*), MPI_Comm (*@\color{red}comm\_cart@*),
    INT (*@\color{green}*local\_n0@*), INT (*@\color{green}*local\_0\_start@*),
    INT (*@\color{orange}*local\_n1@*), INT (*@\color{orange}*local\_1\_start@*));

INT PX(local_size_2d_transposed)(
    INT n0, INT n1, MPI_Comm (*@\color{green}comm@*),
    INT *local_ni0, INT *local_i0_start,
    INT *local_no1, INT *local_o1_start);
INT PX(local_size_2d)(
    INT n0, INT n1, MPI_Comm (*@\color{green}comm@*),
    INT *local_n0, INT *local_0_start);
/* allow arbitrary comm for backward compatibility with FFTW */
\end{lstlisting}


\emph{New PFFT interface: plan (Guru, Adv, Basic, 3d, 2d)}
\begin{lstlisting}
PX(plan) PX(plan_guru_dft)(
    int rnk, const INT *n, const INT *ni, const INT *no,
    const PX(profile) (*@\color{red}*prf@*), const PX(permutation) (*@\color{red}*perm@*),
    INT howmany, const INT *iblock, const INT *oblock,
    C *data_in, C *data_out,
    MPI_Comm comm_cart,
    int sign, unsigned fftw_flags);
/* !!! prf depends on perm !!! */
/* !!! oblock depends on perm, prf, no !!! */

PX(plan) PX(plan_many_dft)(
    int rnk, const INT *n, const INT (*@\color{red}*ni@*), const INT (*@\color{red}*no@*),
    INT (*@\color{red}howmany@*), const INT (*@\color{red}*iblock@*), const INT (*@\color{red}*oblock@*),
    C *data_in, C *data_out,
    MPI_Comm comm_cart,
    int sign, unsigned pfft_flags, unsigned fftw_flags);

PX(plan) PX(plan_dft)(
    int (*@\color{red}rnk@*), const INT (*@\color{red}*n@*),
    C *data_in, C *data_out,
    MPI_Comm comm_cart,
    int sign, unsigned pfft_flags, unsigned fftw_flags);

PX(plan) PX(plan_dft_3d)(
    INT (*@\color{green}n0@*), INT (*@\color{green}n1@*), INT (*@\color{orange}n2@*),
    C *data_in, C *data_out,
    MPI_Comm (*@\color{red}comm\_cart@*),
    int sign, unsigned pfft_flags, unsigned fftw_flags);
PX(plan) PX(plan_dft_2d)(
    INT n0, INT n1, C *data_in, C *data_out,
    MPI_Comm (*@\color{green}comm@*),
    int sign, unsigned pfft_flags, unsigned fftw_flags);
/* allow arbitrary comm for backward compatibility with FFTW */
\end{lstlisting}

\emph{Interface Layer Names (Like FFTW):}
\begin{lstlisting}
local_size[_internal, _guru, _many][_3d, _2d][_transposed]
plan[_internal, _guru, _many]_dft[_3d, _2d]
local_size[_internal, _guru, _many]_gc[_3d, _2d]
plan[_internal, _guru, _many]_gc[_3d, _2d]
\end{lstlisting}

\emph{Interface Layer Names (Like FFTW) - Merged FFT and GC:}
\begin{lstlisting}
local_size[_internal, _guru, _many][_gc][_3d, _2d][_transposed]
plan[_internal, _guru, _many][_dft, _gc][_3d, _2d]
\end{lstlisting}
No need for \lstinline{local_size[_dft, _rdft]}, since we put the physical dimensions into local size.

\newpage
\emph{New PFFT GCell Interface: local size (Basic, Advanced)}
\begin{lstlisting}
INT PX(gc_local_size_internal)(
    int rnk, const INT *loc_array_size, INT tuple_size,
    const INT *gc_below, const INT *gc_above);

INT PX(gc_local_size_many)(
    int rnk, const INT *local_n, const INT *local_start,
    INT alloc_local, INT howmany,
    const INT *num_gcells_below, const INT *num_gcells_above,
    INT *local_ngc, INT *local_gc_start);

INT PX(gc_local_size)(
    int rnk, const INT *local_n, const INT *local_start,
    INT alloc_local,
    const INT *gc_below, const INT *gc_above,
    INT *local_ngc, INT *local_gc_start);
\end{lstlisting}


\emph{New PFFT GCell Interface: plan (Basic, Advanced, Guru, Internal)}
\begin{lstlisting}
PX(gcplan) PX(plan_gcells_internal)(
    int rnk, const INT *n, const PX(permutation) *perm,
    INT howmany, const INT *block,
    const INT *gc_below, const INT *gc_above,
    R *data, MPI_Comm *comms_1d, MPI_Comm comm_cart,
    unsigned gcflags);
// tuple_size = 2*howmany*last_dims

PX(gcplan) PX(plan_guru_gc)(
    int rnk, const INT *n, const PX(permutation) *perm,
    INT howmany, const INT *block,
    const INT *gc_below, const INT *gc_above,
    C *data, MPI_Comm comm_cart, unsigned gcflags);
// gcflags in {ESTIMATE, MEASURE, PATIENT}
//          x {INPLACE, OUTOFPLACE} x {GC_RMA, GC_SENDRECV}
// default: flags = OUTOFPLACE
//          flags += GC_RMA (if possible) otherwise GC_SENDRECV (always possible)
/* !!! block depends on perm !!! */

PX(gcplan) PX(plan_many_gc)(
    int rnk, const INT *n,
    INT howmany, const INT *block,
    const INT *gc_below, const INT *gc_above,
    C *data, MPI_Comm comm_cart, unsigned gcflags);
// gcflags in {NONTRANSPOSED, TRANSPOSED} x {ESTIMATE, MEASURE, PATIENT}
//          x {INPLACE, OUTOFPLACE} x {GC_RMA, GC_SENDRECV}
// default: flags = NONTRANSPOSED

PX(gcplan) PX(plan_gc)(
    int rnk, const INT *n,
    const INT *gc_below, const INT *gc_above,
    C *data, MPI_Comm comm_cart, unsigned gcflags);
// gcflags in {NONTRANSPOSED, TRANSPOSED} x {ESTIMATE, MEASURE, PATIENT}
//          x {INPLACE, OUTOFPLACE} x {GC_RMA, GC_SENDRECV}
\end{lstlisting}


%%%%%%%%%%%%%%%%%%%%%%%%%%%%%%%%%%%%%%%%%%%%%%%%%%%%%%%%%%%%%%%%%%%%%%%%%%%%%%%
\chapter{ToDo}\label{chap:todo}
%%%%%%%%%%%%%%%%%%%%%%%%%%%%%%%%%%%%%%%%%%%%%%%%%%%%%%%%%%%%%%%%%%%%%%%%%%%%%%%

\begin{itemize}
  \item \code{PFFT_FORWARD} is defined as \code{FFTW_FORWARD}
  \item \code{FFTW_FORWARD} is defined as $-1$
  \item PFFT allows to chose between \code{FFTW_FORWARD} and \code{FFTW_BACKWARD}, which is not implemented by FFTW.
  \item Matlab uses the same sign convention, i.e., $-1$ for \code{fft} and $+1$ for \code{ifftn}
\end{itemize}

\section{Pruned FFT and Shifted Index Sets}
\subsection{Pruned FFT}
For pruned r2r- and c2c-FFT are defined as
\begin{equation*}
  g_l = \sum_{k=0}^{n_i-1} \hat g_k \eim{kl/n}, \quad l=0,\hdots,n_o-1,
\end{equation*}
where $n_i\le n$ and $n_o\le n$.

\subsection{Shifted Index Sets}
For $N\in 2\N$ we define the FFT with shifted inputs


For $K,L,N\in 2\N$, $L<N$, $L<N$ we define





\code{FFT_SHIFTED_IN} 



\section{Some Comments on MPI}
Following the MPI standard~\cite{MPI-2.2} we use the term process to denote the smallest MPI processing unit of a parallel, distributed memory machine.
The term process abstracts form the widely used terms processor, node and core.
Note that the term process abstracts the code development from the real physical architecture. For example we can \\

\emph{An MPI program consists of autonomous processes, executing their own code, in an MIMD style.}
( MPI: A Message-Passing Interface Standard, page 27)


