%%%%%%%%%%%%%%%%%%%%%%%%%%%%%%%%%%%%%%%%%%%%%%%%%%%%%%%%%%%%%%%%%%%%%%%%%%%%%%%
\chapter{Installation}\label{chap:inst}
%%%%%%%%%%%%%%%%%%%%%%%%%%%%%%%%%%%%%%%%%%%%%%%%%%%%%%%%%%%%%%%%%%%%%%%%%%%%%%%

The install of PFFT is based on the Autotools and follows the typical workflow
\begin{lstlisting}[escapechar=§]
./configure
make
make install
\end{lstlisting}


\section{Install of the Latest Official FFTW Release}\label{sec:fftw_inst}
PFFT depends on Release~\fftwversion{} of the FFTW library~\cite{fftw}.
For sake of completeness, we show the command line based install procedure in the following.
However, note that we provide install scripts on \websoft that simplify the install a lot.
We highly recommend to use these install scripts, since they additionally apply several
performance patches and bugfixes that have been submitted to the FFTW developers but
are not yet included in the official releases.
\begin{lstlisting}[escapechar=§]
wget http://www.fftw.org/fftw-§\fftwversionsl§.tar.gz
tar xzvf fftw-§\fftwversion§.tar.gz
cd fftw-§\fftwversion§
./configure --enable-mpi --prefix=$HOME/local/fftw3_mpi §\label{lst:fftw:conf}§
make
make install
\end{lstlisting}
The MPI algorithms of FFTW must be build with a MPI C compiler. Add the statement \code{MPICC=\$MPICCOMP}
at the end of line~\ref{lst:fftw:conf} if the \code{configure} script fails to determine the right
MPI C compiler \code{\$MPICCOMP}. Similarly, the MPI Fortran compiler \code{\$MPIFCOMP} is set by \code{MPIFC=\$MPIFCOMP}.

\section{Install of the PFFT library}
In the simplest case, the hardware platform and the \fftw-\fftwversion{} library are
recognized by the PFFT configure script automatically, so all we have to do is
\begin{lstlisting}[escapechar=§]
wget http://www.tu-chemnitz.de/~mpip/software/pfft-§\pfftversionsl§.tar.gz
tar xzvf pfft-§\pfftversion§.tar.gz
cd pfft-§\pfftversion§
./configure
make
make check
make install
\end{lstlisting}
Hereby, the optional call \code{make check} builds the test programs.
If the \fftw-\fftwversion{} software library is already installed on your system but not found by the configure script,
you can provide the FFTW installation directory \code{\$FFTWDIR} to configure by
\begin{lstlisting}
./configure --with-fftw3=$FFTWDIR
\end{lstlisting}
This call implies that the FFTW header files are located in \code{\$FFTWDIR/include} and the FFTW library files are located
in \code{\$FFTWDIR/lib}. Otherwise, one should specify the FFTW include path \code{\$FFTWINC} and the FFTW library path
\code{\$FFTWLIB} separately by
\begin{lstlisting}
./configure --with-fftw3-includedir=$FFTWINC --with-fftw3-libdir=$FFTWLIB
\end{lstlisting}
To install PFFT in a user specified directory \code{\$PFFTINSTDIR} call configure with the option
\begin{lstlisting}
./configure --prefix=$PFFTINSTDIR
\end{lstlisting}
This mandatory whenever you do not have root permissions on your machine, since the default install paths of 
\code{configure} are not accessible by normal users.
The PFFT library must be build with a MPI compiler. In Section~\ref{sec:fftw_inst} we describe how to hand the right compilers to the \code{configure} script.
For more details on the options of the \code{configure} script call
\begin{lstlisting}
./configure --help
\end{lstlisting}
