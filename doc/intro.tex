This user manual describes the usage of PFFT~\pfftversion~\cite{pfft,Pi13}, a MPI-based, parallel software library for the
computation of equispaced fast Fourier transforms (FFT) on parallel, distributed memory architectures.

%%%%%%%%%%%%%%%%%%%%%%%%%%%%%%%%%%%%%%%%%%%%%%%%%%%%%%%%%%%%%%%%%%%%%%%%%%%%%%%
\chapter{Introduction}\label{chap:intro}
%%%%%%%%%%%%%%%%%%%%%%%%%%%%%%%%%%%%%%%%%%%%%%%%%%%%%%%%%%%%%%%%%%%%%%%%%%%%%%%
A popular software library for computing FFTs is FFTW~\cite{fftw, FFTW05}. This library also includes a parallel FFT implementation (FFTW-MPI) based on the Message Passing Interface (MPI).
FFTW-MPI parallelizes multi-dimensional FFTs by a mixture of serial lower-dimensional FFTs and parallel data transpositions.
However, FFTW-MPI makes use of a one-dimensional data decomposition, which shows to be a scalability bottleneck on large scale, parallel computers.
For example, a three-dimensional FFT of size $1024^3$ can be computed with at most $1024$ MPI processes.
In contrast, using a two-dimensional data decomposition would increase the maximum number of MPI processes to $1024^2$ in this case.

The main goal of PFFT is to extend the MPI part of the FFTW software library to multi-dimensional data decompositions,
i.e., $d$-dimensional FFTs of size $N^d$ can be computed in parallel with at most $N^{d-1}$ MPI processes.
In addition, PFFT offers several extra features that are particular usefull for parallel, distributed memory FFTs but are not yet present in FFTW-MPI.
The interface of PFFT is as close as possible to the FFTW-MPI interface. Porting code that uses FFTW-MPI to PFFT is almost trivial.
We refer to the publication~\cite{Pi13} for a closer look on the different data decompositions and the underlying algorithms of the PFFT library.

Most features of PFFT are inherited from FFTW or similarily implemented. These include the following:
\begin{compactitem}
  \item We employ fast $\mathcal{O}(N\log N)$ algorithms of FFTW to compute arbitrary-size
        discrete Fourier transforms of complex data, real data, and even- or odd-symmetric real data.
  \item The dimension of the FFT can be arbitrary. However, parallel data decomposition must be at least one dimension smaller.
  \item PFFT offers portable performance; e.g., it will perform well on most platforms.
  \item The application of PFFT is split into a time consuming planning step and a high performance execution step.
  \item Installing the library is easy. It is based on the common sequence of configure, make, and make install.
  \item The interface of PFFT is very close to the MPI interface of FFTW.
        In fact, we tried to add as few extra parameters as possible.
  \item PFFT is written in C but also offers a Fortran interface, see Section~\ref{sec:fortran}.
  \item FFTW includes shared memory parallelism for all serial transforms. This enables us to benefit from hybrid parallelism to a certain amount, see Section~\ref{sec:openmp}.
  \item All steps of our parallel FFT can be performed completely in place. This is especially remarkable for the global
        transposition routines.
  \item Confirming to good MPI programming practice, all PFFT transforms can be performed on user defined communicators.
        In other words, PFFT does not enforce the user to work with \verb+MPI_COMM_WORLD+.
  \item PFFT uses the same algorithm to compute the size of the local array blocks as FFTW. This implies that the FFT size need not
        be divisible by the number of processes.
  \item PFFT supports single, double and long double precision.
  \item PFFT supports new-array execution, i.e., a PFFT plan can be planned and executed on different plans up to some restrictions, see Section~\ref{sec:new-array} for details.
        Thanks to Yu Feng for the new-array execute patch.
\end{compactitem}
Furthermore, we added some special features to support repeated tasks that often occur in practical application of parallel FFTs.
\begin{compactitem}
  \item PFFT includes a very flexible ghost cell exchange module. A detailed description of this module is given in Section~\ref{sec:ghost}.
  \item PFFT accepts three-dimensional data decomposition even for three-dimen\-sional FFTs.
        However, the underlying parallel FFT framework is still based on two-dimensional decomposition. A more detailed description can be found
        in Section~\ref{sec:remap3dto2d}.
  \item PFFT explicitly supports the parallel calculation of pruned FFTs. Details are given in Section~\ref{sec:pruned}.
\end{compactitem}

Finally, we complete this overview with a list of features that are (not yet) implemented in PFFT.
\begin{compactitem}
  \item Parallel one-dimensional FFT based on MPI. FFTW-MPI uses another parallelization strategy for one-dimensional FFTs, which is not implemented in PFFT.
        The reason is that we can achive a scalability benefit since all dimensions of the FFT are already distributed.
  \item PFFT does not have full OpenMP support. All serial FFT computations and global communications are implemented with FFTW,
        which offers OpenMP support. See Section~\ref{sec:openmp}. However, most of the PFFT-only features, such as pruned FFT, ghost cell send and 3d decompostion of 3d FFTs are not yet parallelized with OpenMP.
  \item PFFT does not overlap communication and computation. The code of PFFT is build in a very modularized structure. Most of these modules consist
        of FFTWs routines. Therefore, the global transposition does not support non blocking communication.
  \item We do not provide any parallel IO routines. The user is responsible for the load and store of parallel data.
  \item PFFT depends on FFTW to perform its serial transforms. Since we use some of the most fancy features of FFTW, the serial transforms
        could only be replaced partly by other FFT vendors. However, this is not a big drawback, since FFTW seems to perform very well on most platforms.
  \item The global communication routines can not be called separately. However, it should be possible to implement a user interface to our global
        transposition routines.
  \item PFFT does not support GPU parallelization.
\end{compactitem}
You are welcome to add new PFFT features at \webpfft.

The reader of this manual should familiar with the basic usage of FFTW and MPI.
For further information we refer to the well written FFTW user manual~\cite{fftw-manual} and
the MPI Standard~\cite{MPI-2.2}, see also \cite{GrLuTh99} for detailed explanations.

\section{Alternative parallel FFT implementations}
There have been several FFT implementations that aim to circumvent the scalability bottleneck
for at least three dimensional FFTs by using two-dimensional decomposition approach.
However, these implementations are often fitted to special problems and where not published
as a stand alone software library. 
Remarkable exceptions are the parallel FFT software library by S.~Plimpton~\cite{Pl97,sandiafft},
the P3DFFT software library by D.~Pekurovsky~\cite{Pe12,p3dfft} and the \mbox{2DECOMP\&FFT} software library by N.~Li~\cite{Li2010, 2decompfft}.

\section{Parallel nonequispaced FFT}
If your are interested in a parallel implementation of nonequispaced fast Fourier
transforms (NFFT) for distributed memory architectures, you should have a look at our PNFFT software library~\cite{pnfft, PiPo13}
that is also available at \webpnfft.

