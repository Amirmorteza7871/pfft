%%%%%%%%%%%%%%%%%%%%%%%%%%%%%%%%%%%%%%%%%%%%%%%%%%%%%%%%%%%%%%%%%%%%%%%%%%%%%%%
\chapter{Introduction}\label{chap:intro}
%%%%%%%%%%%%%%%%%%%%%%%%%%%%%%%%%%%%%%%%%%%%%%%%%%%%%%%%%%%%%%%%%%%%%%%%%%%%%%%
This user manual describes the usage of PFFT~\pfftversion, a parallel software library for the
computation of equispaced fast Fourier transforms on parallel, distributed memory architectures.
Although the well known FFTW software library~\cite{fftw, FFTW05} already supports parallel FFT
on distributed memory architectures, its one-dimensional data decomposition limits the usage on
massively parallel machines. PFFT can be understood as a generalization of the MPI algorithms
from FFTW to data decompositions of arbitrary dimension.
We refer to the publication~\cite{Pi13} for a closer look on the
different data decompositions and the underlying algorithms of the PFFT library.



We suppose that the reader is at least familiar with basic usage of FFTW. Since our implementation is based on
MPI, the reader should also know about the basics of distributed memory parallelism and the usage of MPI for parallel
programming.




\section{Comparison with other parallel FFT implementations}

There have been several FFT implementations that aim to circumvent the scalability bottle neck
for at least three dimensional input data by using two-dimensional decomposition approach.
Nevertheless these implementations are often fitted to special problems and where not published
as a stand alone software library. Remarkable exceptions are the S.~Plimpton~\cite{libsandiafft},
the P3DFFT~\cite{libp3dfft} software library by D.~Pekurovsky and the \mbox{2DECOMP\&FFT} software library~\cite{Li2010, lib2decompfft}.
% \cite{Pi2012, pfft}



\section{Why to implement parallel FFT again?}
As there are several software libraries that deal with the scalability bottleneck, one could ask why we implemented the two-dimensional data decomposition again.
The answer lies in the usability and flexibility of the available software libraries in comparison to FFTW.
Our vision is to implement a parallel FFT library, that adepts to the FFTW interface as much as possible
and preserves most of the nice features that FFTW provides, e.g, high portability, self adapting runtime optimization and in place transforms.

\section{Comparision to P3DFFT}
In this section we compare the features and runtimes of PFFT to the P3DFFT software library.
To get a fair comparison, the P3DFFT feature list has to be up to date. If there is any feature missing, please contact us.

\section{About the name PFFT}
Since we are not the first people who came up with a parallel FFT implementation, it was hard
to find a short and unique abbreviation for our library. To append a single P (for parallel)
to the FFT seems to be very natural and fits perfect in our naming conventions for generalized
FFT, i.e., the N for nonequispaced in the abbreviation NFFT or an S for spherical in NSFFT and so on.
As one can imagine we are not the first who came up with this idea. Therefore, we apologize
to all the libraries, that already used the same name. Especially the first hit on most search engines
leads to the PFFT library ... As this library seems to be outdated and does not work an any current
machine we decided to reuse its name.

If you have concerns that the PFFT name struggles your interests, please contact us.


One could argue, that PFFT sounds very general and therefore is somehow \emph{anma\ss{}end}.
That's true and we are aware of this


The need of a highly scalable parallel FFT software library can not be over estimated. 

PFFT uses a two-dimensional data decomposition.
PFFT aims to provide a highly scalable FFT implementation

\begin{compactitem}
  \item PFFT is based on FFTW \cite{fftw} and therefore provides most of the features of FFTW
        (arbitrary size DFT in $\Cal{O}(n \log n)$, )
  \item PFFT computes the DFT of complex data
  \item The input data can have arbitrary length.
  \item The number of processes 
\end{compactitem}


\begin{compactitem}
  \item 2d data decomposition if possible (at least 3dFFT)
  \item tries to get same performance as FFTW for the other cases (in fact we only call a wrapper to FFTW)
  \item c2c FFTs
\end{compactitem}

Summarize sections:
\begin{compactitem}
  \item Section 1: Tutorial
\end{compactitem}


If your are interested in a parallel implementation of nonequispaced fast Fourier
transforms, you should have a look at the PNFFT software library. It is also available
at \url{http://www.tu-chemnitz.de/~mpip/software}.


MPI strictly distinguishes between processes and processors. 

%------------------------------------------------------------------------------
\section{something}
%------------------------------------------------------------------------------
some text
